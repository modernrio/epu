\chapter{Befehlssatzarchitektur}
\label{c:befehlssatzarchitektur}
\section{Grundgedanken}
Die Befehlssatzarchitektur beschreibt die Schnittstelle zwischen Hardwaredesign und
Anwendungssoftware. Da die Veränderungen der Befehlssatzarchitektur Einfluss sowohl auf das
Hardwaredesign als auch die Anwendungssoftware hat, wird diese hier zuerst beschrieben.
Das Endergebnis der Befehlssatzarchitektur ist ein auf den Rechner angepassten Befehlssatz,
welcher die Hardware mit der Software verbindet. Das Hauptziel des Befehlssatzes der EPU ist es,
möglichst viele elemantere Operationen mit möglichst wenigen Befehlen auszuführen. Dabei wird auch
auf einen einfachen Hardwareaufbau zur Implementierung des Befehlssatzes geachtet, um die Anzahl
der genutzten Logikeinheiten zu reduzieren. Über die Anzahl der Logikeinheiten wird in
Kapitel~\ref{c:hardwaredesign} genaueres beschrieben.
\subsection{Registerbelegung}
Die \ac{EPU} besitzt 16 Register, welche durch Selektion von $\log_2(16) = 4$ Adressbits
angesprochen werden. Mithilfe der Tabelle~\ref{tab:registerbelegung} soll
eine Übersicht aller Register dargestellt werden.

\begin{table}[h]
\centering
\begin{tabular}{lll}
\toprule
Selektion & Name & Zweck\\
\midrule
0000 & R0  & Akkumulator\\
0001 & R1  & Allgemeine Verwendung\\
0010 & R2  & Laufvariable\\
0011 & R3  & Datenregister\\
0100 & R4  & Allgemeine Verwendung\\
0101 & R5  & Allgemeine Verwendung\\
0110 & R6  & Allgemeine Verwendung\\
0111 & R7  & Allgemeine Verwendung\\
1000 & R8  & Allgemeine Verwendung\\
1001 & R9  & Allgemeine Verwendung\\
1010 & R10 & Allgemeine Verwendung\\
1010 & R11 & Allgemeine Verwendung\\
1100 & R12 & Allgemeine Verwendung\\
1101 & R13 & Allgemeine Verwendung\\
1110 & FLA & Flagregister\\
1111 & ID  & Interruptdaten Verwendung\\
\bottomrule
\end{tabular}
\caption{Registerbelegung}
\label{tab:registerbelegung}
\end{table}
\clearpage
\section{Befehlsformen}
