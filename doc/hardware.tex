\chapter{Hardware}
\label{c:hardwar}
\section{Aufbau}
\subsection{Steuerwerk}
\pagebreak
\subsection{Rechenwerk}
\pagebreak
\subsection{Dekodierer}
\pagebreak
\subsection{Programmzähler}
\pagebreak
\subsection{Stack}
\pagebreak
\subsection{Arbeitsspeicher}
\pagebreak
\subsection{Registerbelegung}
Die \ac{EPU} besitzt 16 Register, welche durch Selektion von $\log_2(16) = 4$
Adressbits angesprochen werden. Mithilfe der Tabelle~\ref{tab:registerbelegung}
soll eine Übersicht aller Register dargestellt werden.

\begin{table}[h]
\centering
\begin{tabular}{lll}
\toprule
Selektion & Name & Zweck\\
\midrule
0000 & R0  & Akkumulator\\
0001 & R1  & Allgemeine Verwendung\\
0010 & R2  & Laufvariable\\
0011 & R3  & Datenregister\\
0100 & R4  & Allgemeine Verwendung\\
0101 & R5  & Allgemeine Verwendung\\
0110 & R6  & Allgemeine Verwendung\\
0111 & R7  & Allgemeine Verwendung\\
1000 & R8  & Allgemeine Verwendung\\
1001 & R9  & Allgemeine Verwendung\\
1010 & R10 & Allgemeine Verwendung\\
1010 & R11 & Allgemeine Verwendung\\
1100 & R12 & Allgemeine Verwendung\\
1101 & R13 & Allgemeine Verwendung\\
1110 & FLA & Flagregister\\
1111 & ID  & Interruptdaten\\
\bottomrule
\end{tabular}
\caption{Registerbelegung}
\label{tab:registerbelegung}
\end{table}
\section{Befehlssatz}
\section{Ein- und Ausgabe}
\subsection{Eingabe}
\subsection{Ausgabe}
