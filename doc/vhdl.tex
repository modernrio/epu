\chapter{Implementierung in VHDL}
\label{c:vhdl}
\section{Einführung in VHDL}
In diesem Abschnitt soll eine kurze Einführung in VHDL gegeben werden, damit die
nachfolgende Erklärung der Implementierung leichter zu verstehen ist.

\subsection{Was ist VHDL}
VHDL ist eine sogenannte Hardware Description Language (HDL), mit welcher es möglich ist, digitale
Schaltungen in Form von Quellcode zu schreiben, um diesen dann von einem
Computer zu Hardware weiterverarbeiten zu lassen. VHDL ist eine Abkürzung für
VHSIC HDL (\acl{VHDL}). Der geschriebene VHDL-Code wird dann synthesiert zu
einer Netzliste, welche die einzelnen Verbindungen zwischen den Komponenten
beschreibt. Diese Netzliste wird daraufhin für den jeweiligen \ac{FPGA}
übersetzt, sodass als letzter Schritt eine Konfigurationsdatei erstellt werden
kann, welche auf den \ac{FPGA} geladen werden kann.

Eine Besonderheit bei VHDL ist, dass nicht jeder geschriebene Quellcode
synthesierbar ist, was bedeutet, dass mancher Quellcode nur in der Simulation
funktioniert. In dem Umfang dieses Projektes ist aber der geschriebene Quellcode
vollkommen synthesierbar und soll nicht nur in der Simulation funktionieren.

\subsection{Notation}
Allgemein endet jede Anweisung immer mit einem Semikolon. Kommentare werden mit
einem doppelten Bindestrich (-{}-) eingeleitet und sind gültig bis zum Ende der
Zeile. Einzelne Bitwerte werden mit einfachen Anführungszeichen und Werte
von Bitvektoren (mehreren Bits) mit doppelten Anführungszeichen umrahmt.

Außerdem gibt es noch mehrere Sprachelemente, welche häufig Verwendung
haben. Eine (nicht vollständige) Beschreibung von häufig genutzten
Sprachelementen:

\begin{description}[align=right, labelwidth=1.6cm]
\item[signal]		Verbindung zweier Module. Beschreibt die Leitung zwischen den
					Modulen.
\item[variable]		Ein Zwischenspeicher für Werte, welcher \textbf{nicht} als
					Signal übersetzt wird.
\item[entity]		Schlüsselwort zur Deklaration eines Moduls (logische Einheit).
\item[in]			Deklaration eines Eingabesignals eines Moduls.
\item[out]			Deklaration eines Ausgabesignals eines Moduls.
\item[architecture]	Schlüsselwort zur Definition eines Moduls
\item[component]	Signatur eines Moduls. Wird verwendet, um ein Modul
					innerhalb eines anderen Moduls zu verwenden.
\item[package]		Eine Sammlung (Bibliothek) von \textbf{components}.
\end{description}
\pagebreak
\section{Beschreibung wichtiger Module}
\subsection{top -- Verbindung der Ein- und Ausgänge}
Das Topmodul dient als oberste Ebene, welche die \ac{EPU} mit der `Außenwelt'
verbindet. Die dazugehörigen Ein- und Ausgangssignale werden in der Datei
\textit{top.vhdl} festgelegt. Abbildung~\ref{code:top} zeigt die Definition des
Topmoduls.
\begin{figure}[htb]
\lstinputlisting[language=vhdl, tabsize=4, firstline=10, lastline=29]{../vhdl/top/top.vhdl}
\caption{Topmodul}
\label{code:top}
\end{figure}
\subsection{core -- Topmodul der CPU}
\pagebreak
\subsection{memory\_control -- Speichercontroller}
\pagebreak
