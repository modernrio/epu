\chapter{Einleitung}
\label{c:einleitung}
Diese Dokumentation beschreibt den Aufbau und die Funktionsweise der \ac{EPU}.
Das Projekt kam dadurch zustande, dass die Struktur und die Arbeitsweise eines
Computers, insbesondere einer \ac{CPU} besser verstanden werden soll. Um dieses
Ziel zu erreichen, wurde die \ac{EPU} erstellt, da sie als lehrreicher
Mikrorechner die Funktionsweise und den Aufbau eines Alltagscomputer erklärt und
somit Verständnis für die Komplexität unserer heutigen Rechner einbringt.

\section{Was ist ein FPGA?}
Ein \ac{FPGA} ist ein \ac{IC}, welcher zum Aufbau digitaler Schaltungen
dient. Er besteht meist aus mehr als 100.000 Logikblöcken~\cite[S. 8]{minicpu}.
Ins Deutsche übersetzt bedeutet \ac{FPGA} soviel wie "`im Feld programmierbare
(Logik-)Gatter-Anordnung"'.

Die Logikschaltungen eines \ac{FPGA} sind entweder über elektronische "`Schalter"'
der Konfiguration entsprechend verknüpft oder es werden sogenannte \acp{LUT}
benutzt, mit denen die Logikfunktion explizit realisiert werden kann. Eine
\ac{LUT} kann verschiedene kombinatorische Funktionen (NAND, XOR, AND, NOT,
Multiplexer, etc.) aus den Eingangssignalen realisieren. Die meisten \acp{LUT}
besitzen zwischen 4 und 6 Eingangssignale. Es ist auch möglich, mehrere
\acp{LUT} in Serie zu schalten, wodurch die Limitierung der Eingangssignale
entfernt werden kann.~\cite{FPGA_Aufbau}

Da der \ac{FPGA} nach Verlust des Stromanschlusses die Konfiguration der
Logikelemente nicht von selbst speichert, wird meist zusätzlich noch ein
Flash-Speicher verbaut, damit nach einem Stromverlust die alte Konfiguration
wieder neu geladen werden kann. Dies hat auch den Vorteil, dass dadurch der
Status des \ac{FPGA} zurückgesetzt wird und somit ein "`Neustart"' schnell möglich
ist.

Anders als bei üblicher Programmierung von Computern kann bei einem \ac{FPGA}
keine herkömmliche Programmiersprache verwendet werden. Das "`Programmieren"' wird
üblicherweise als Konfiguration bezeichnet und wird mithilfe einer
Hardwarebeschreibungssprache wie z.B. VHDL oder Verilog erledigt. Auch wird der
geschriebene Code nicht kompiliert, wie bei der herkömmlichen Programmierung,
sondern synthetisiert, was soviel bedeutet wie "`künstlich herstellen"'.
\section{Beschreibung des genutzten FPGA}
Vor der Suche nach einem FPGA-Board waren bereits einige Anforderungen klar,
welche das Board erfüllen muss:
\begin{itemize}
	\item Videoausgang
	\item Konfiguration über USB
	\item \acs{UART} (Serielle Schnittstelle)
	\item 64KiB Speicher
\end{itemize}
Nach etwas längerer Suche wurde das "`Waxwing Spartan 6 FPGA Development Board"'
von Numato Lab ausgewählt, da es alle Anforderungen auf einem Entwicklungsboard
vereint. Einige Merkmale dieses Entwicklungsboards sind:
\begin{itemize}
	\item Xilinx Spartan 6 FPGA
	\item Taktrate 100MHz
	\item Videoausgang (HDMI und VGA)
	\item LCD-Display
	\item Drei 7-Segment-Anzeigen
	\item USB-UART
	\item MicroSD-Support
	\item Größe: 180mm x 120mm
\end{itemize}
