% Dokumentenklasse
\documentclass[a4paper,12pt,liststotoc, parskip=half]{scrreprt}

% ============ Pakete ============
% Dokumentinformationen
\usepackage[
	pdftitle={Dokumentation EPU},	
	pdfsubject={},
	pdfauthor={Markus Schneider},
	pdfkeywords={},
	% Links nicht einrahmen
	hidelinks
]{hyperref}

% Standardpakete
\usepackage[utf8]{inputenc}				% UTF-8 Zeichensatz
\usepackage[ngerman]{babel}				% Alle Bezeichnungen auf die deutsche Sprache anpassen
\usepackage[T1]{fontenc}				% Unterstützung für westeuropäische Codierung(Umlaute)
\usepackage{graphicx}					% Grafiken einbinden
\usepackage{subfig}						% Abbildungen und Tabellen
\graphicspath{{img/}}					% Pfad zu Grafiken
\usepackage{fancyhdr}					% Einfache Bearbeitung von Kopf- und Fußzeile
\usepackage{lmodern}					% Verändert die Schriftart auf "Latin Modern"
\usepackage{color}						% Farbenmanagement
\usepackage[printonlyused]{acronym}		% Abkürzungsverzeichnis
\usepackage{booktabs}					% Tabellen ("Publication Quality")
\usepackage{setspace}					% Abstände
\usepackage{pgf}						% Makropaket zum Erstellen von Grafiken
\usepackage{tikz}						% Vektorgrafiksprache
\usepackage{bm}							% Fettschrift für mathematische Objekte
\usepackage{mathptmx}					% Adobe Times Roman als Schriftart
\usepackage{float}						% Verbessert die Schnittstelle für Gleitobjekte
\usepackage{enumitem}					% Kontrolle über das Layout von Auflistungen
\usepackage{listingsutf8}				% Quellcode
\usepackage{todonotes}					% TODO-Liste

% Quellcodeumlaute
\lstset{literate=
  {á}{{\'a}}1 {é}{{\'e}}1 {í}{{\'i}}1 {ó}{{\'o}}1 {ú}{{\'u}}1
  {Á}{{\'A}}1 {É}{{\'E}}1 {Í}{{\'I}}1 {Ó}{{\'O}}1 {Ú}{{\'U}}1
  {à}{{\`a}}1 {è}{{\`e}}1 {ì}{{\`i}}1 {ò}{{\`o}}1 {ù}{{\`u}}1
  {À}{{\`A}}1 {È}{{\'E}}1 {Ì}{{\`I}}1 {Ò}{{\`O}}1 {Ù}{{\`U}}1
  {ä}{{\"a}}1 {ë}{{\"e}}1 {ï}{{\"i}}1 {ö}{{\"o}}1 {ü}{{\"u}}1
  {Ä}{{\"A}}1 {Ë}{{\"E}}1 {Ï}{{\"I}}1 {Ö}{{\"O}}1 {Ü}{{\"U}}1
  {â}{{\^a}}1 {ê}{{\^e}}1 {î}{{\^i}}1 {ô}{{\^o}}1 {û}{{\^u}}1
  {Â}{{\^A}}1 {Ê}{{\^E}}1 {Î}{{\^I}}1 {Ô}{{\^O}}1 {Û}{{\^U}}1
  {œ}{{\oe}}1 {Œ}{{\OE}}1 {æ}{{\ae}}1 {Æ}{{\AE}}1 {ß}{{\ss}}1
  {ű}{{\H{u}}}1 {Ű}{{\H{U}}}1 {ő}{{\H{o}}}1 {Ő}{{\H{O}}}1
  {ç}{{\c c}}1 {Ç}{{\c C}}1 {ø}{{\o}}1 {å}{{\r a}}1 {Å}{{\r A}}1
  {€}{{\euro}}1 {£}{{\pounds}}1 {«}{{\guillemotleft}}1
  {»}{{\guillemotright}}1 {ñ}{{\~n}}1 {Ñ}{{\~N}}1 {¿}{{?`}}1
}

% Quellcodefarben
\definecolor{codegreen}{rgb}{0,0.6,0}
\definecolor{codegray}{rgb}{0.5,0.5,0.5}
\definecolor{codemauve}{rgb}{0.58,0,0.82}

% Quellcodeoptionen
\lstset{
	basicstyle=\footnotesize,
  	breakatwhitespace=false,
  	breaklines=true,
  	captionpos=b,
  	commentstyle=\color{codegreen},
  	extendedchars=true,
  	frame=none,
	keepspaces=true,
	columns=flexible,
  	keywordstyle=\color{blue},
	language=vhdl,
  	numbers=left,
  	numbersep=5pt,
  	numberstyle=\color{codegray},
  	rulecolor=\color{black},
  	showspaces=false,
  	showstringspaces=false,
  	showtabs=false,
	stepnumber=1,
  	stringstyle=\color{codemauve},
	tabsize=4,
	% title=\lstname,
	backgroundcolor=\color{white}
}

% Vektorgrafikbibliotheken
\usetikzlibrary{arrows,automata,decorations.pathmorphing,backgrounds,fit,positioning,shapes.symbols,chains,shapes.geometric,shapes.arrows,calc}

% Zusätzliche Schriftzeichen der American Mathematical Society
\usepackage{amsfonts}
\usepackage{amsmath}

% Zusätzliche Befehle
\newcommand{\mx}[1]{\mathbf{\bm{#1}}} % Matrix command
\newcommand{\vc}[1]{\mathbf{\bm{#1}}} % Vector command
\newcommand{\titledate}[2][2.5in]{%
  \noindent%
  \begin{tabular}{@{}p{#1}@{}}
    \\ \hline \\[-.75\normalbaselineskip]
    #2
  \end{tabular} \hspace{1in}
  \begin{tabular}{@{}p{#1}@{}}
    \\ \hline \\[-.75\normalbaselineskip]
    Datum
  \end{tabular}
}

% ============ Kopf- und Fußzeile ============
% Style
\pagestyle{fancy}

% Kopfzeile
\lhead{}
\chead{}
\rhead{\slshape \leftmark}

% Fußzeile
\lfoot{}
\cfoot{\thepage}
\rfoot{}

% Trennlinien
\renewcommand{\headrulewidth}{0.4pt}	% Dünne Trennline nach der Kopfzeile
\renewcommand{\footrulewidth}{0pt}		% Keine Trennline vor der Fußzeile

% ============ Paketeinstellungen & Sonstiges ============
% Besondere Trennungen
\hyphenation{De-zi-mal-tren-nung}

% ============ Dokumentbeginn ============
\begin{document}

% Seiten ohne Kopf- und Fußzeilen sowie Seitenzahl
\pagestyle{empty}

% Festlegung der Art der Zitierung
\bibliographystyle{unsrt}

\title{Implementierung eines Mikrorechners in VHDL auf einem FPGA}
\date{\today}
\author{Markus Schneider}

\maketitle


% Beendet eine Seite und erzwingt auf den nachfolgenden Seiten die Ausgabe aller Gleitobjekte
% (z.B. Abbildungen), die bislang definiert, aber noch nicht ausgegeben wurden. Dieser Befehl fügt,
% falls nötig, eine leere Seite ein, sodaß die nächste Seite nach den Gleitobjekten eine ungerade
% Seitennummer hat. 
\cleardoubleoddpage{}

% Seiten ab jetzt mit Kopf- und Fußzeilen sowie Seitenzahl
\pagestyle{fancy}

% Inhaltsverzeichnis
\tableofcontents

% Abkürzungsverzeichnis
\begin{acronym}[VHDL]

\thispagestyle{empty}
\chapter*{Abkürzungsverzeichnis}
\addcontentsline{toc}{chapter}{Abkürzungsverzeichnis}
\acro{EPU}{Educational Processing Unit}
\acro{CPU}{Central Processing Unit}
\acro{FPGA}{Field-programmable gate array}
\acro{IC}{Integrated Circuit}
\acro{LUT}{Look-Up-Table}
\acrodefplural{LUT}[LUTs]{Look-Up-Tables}
\acro{UART}{Universal Asynchronous Receiver Transmitter}
\acro{ALU}{Arithmetic logic unit}
\acro{PC}{Program Counter}
\acro{VHDL}{\textbf{V}ery High Speed Integrated Circuit \textbf{H}ardware
	\textbf{D}escription \textbf{L}anguage}

\end{acronym}


% Abbildungsverzeichnis
\listoffigures

% Tabellenverzeichnis
\listoftables

% Selbstständigkeitserklärung
\thispagestyle{empty}
\chapter*{Selbstständigkeitserklärung}
\addcontentsline{toc}{chapter}{Selbstständigkeitserklärung}
 
Hiermit versichere ich, dass die vorliegende besondere Lernleistung
selbstständig und nur unter Verwendung der angegebenen Quellen und Hilfsmitteln
angefertigt wurde. Mir ist bekannt, dass bei nachgewiesenen Täuschungsversuchen
die Prüfung als „nicht bestanden“ erklärt werden kann. 
 
\vspace{1cm}
\titledate{Unterschrift Autor}


% Kapitel einbinden
\chapter{Einleitung}
\label{c:einleitung}
Diese Dokumentation beschreibt den Aufbau und die Funktionsweise der \ac{EPU}.
Das Projekt kam dadurch zustande, dass die Struktur und die Arbeitsweise eines
Computers, insbesondere der \ac{CPU} besser verstanden werden soll. Um dieses
Ziel zu erreichen, wurde die \ac{EPU} gebaut, da sie als lehrreicher
Mikrorechner, wobei der Hauptteil der \ac{EPU} nur aus der \ac{CPU} besteht,
die Funktionsweise und den Aufbau eines Alltagscomputer erklärt und somit
Verständnis für die Komplexität unserer heutingen Rechner einbringt.

\section{Was ist ein FPGA?}
Ein \ac{FPGA} ist ein \ac{IC}, welcher zum Aufbau digitaler Schaltungen
dient. Er besteht meist aus mehr als 100.000 Logikblöcken~\cite[S. 8]{minicpu}.
Ins Deutsche übersetzt bedeutet \ac{FPGA} soviel wie `im Feld programmierbare
[Logik-]Gatter-Anordnung', wobei `im Feld' sich dabei auf den Konsumenten
bezieht.

Die Logikschaltungen eines \ac{FPGA} sind entweder über elektronische `Schalter'
der Konfiguration entsprechend verknüpft oder es werden sogenannte \acp{LUT}
benutzt, mit denen die Logikfunktion explizit realisiert werden kann. Eine
\ac{LUT} kann verschiedene kombinatorische Funktionen (NAND, XOR, AND, NOT,
Multiplexer, etc.) aus den Eingangssignalen realisieren. Die meisten \acp{LUT}
besitzen zwischen 4 und 6 Eingangssignale. Es ist auch möglich, mehrere
\acp{LUT} in Serie zu schalten und die Limitierung durch die Eingangssignale zu
verhindern~\cite{FPGA_Aufbau}.

Da der \ac{FPGA} aber nach Verlust des Stromanschlusses die Konfiguration der
Logikelemente nicht von selbst speichert, wird meist zusätzlich noch ein
Flash-Speicher verbaut, damit nach einem Stromverlust die alte Konfiguration
wieder neu geladen werden kann. Dies hat auch den Vorteil, dass dadurch der
Status des \ac{FPGA} zurückgesetzt wird und somit ein `Neustart' schnell möglich
ist.

Anders als bei üblicher Programmierung von Computern kann bei einem \ac{FPGA}
keine herkömmliche Programmiersprache verwendet. Das `Programmieren' wird
üblicherweise als Konfiguration bezeichnet und wird mithilfe einer
Hardwarebeschreibungssprache wie z.B. VHDL oder Verilog erledigt.
\section{Beschreibung des genutzten FPGA und Entwicklungsboards}
\section{Einführung in VHDL}

\chapter{Übersicht der Hardwarekomponenten}
\label{c:hardware}
\section{Komponentenbeschreibung}
Es erfolgt eine Übersicht aller Hardwarekomponenten.  Dadurch wird eine grobe
Vorstellung der Funktionsweise des Mikrorechners entstehen, welche das
Kapitel~\ref{c:vhdl} vertieft. Vor der Beschreibung der einzelnen
Hardwarekomponenten soll durch Abbildung~\ref{pic:hardware_overview} das
Zusammenspiel aller Komponenten gezeigt werden, auf welche während dieses
Kapitels zurückgegriffen werden kann.
\begin{figure}[hb]
\centering
\def\svgwidth{\columnwidth}
\input{img/hardware_overview.pdf_tex}
\caption{Hardwarekomponentenübersicht}
\label{pic:hardware_overview}
\end{figure}
\pagebreak
\subsection{Steuerwerk}
\label{s:control}
Das Steuerwerk ist dafür zuständig, die einzelnen Zustände, die bei der
Ausführung jedes Befehls ausgeführt werden, zum richtigen Zeitpunkt zu
aktivieren. Dabei können einzelne Zustände je nach Befehl übersprungen werden.

Als Abstraktion kann man sich das Steuerwerk auch als Zustandsautomat
vorstellen. Mit Hilfe von Abbildung~\ref{pic:zustandsdiagramm} ist die
Funktionsweise des Steuerwerkes durch ein Zustandsdiagramm dargestellt.

\begin{figure}[htb]
\centering
\begin{tikzpicture}[->,>=stealth',shorten >=1pt,auto,node distance=4cm,
					semithick]

	\tikzstyle{every state}=[fill=white,draw=black,text=black,minimum size=6em]

	\node[initial,state]	(A)						{Fetch};
	\node[state]			(B) [above right of=A]	{Decode};
	\node[state]			(C) [right of=B]		{Read};
	\node[state]			(D) [below of=C]		{Execute};
	\node[state]			(F) [below of=D]		{Stack R/W};
	\node[state]			(E) [right of=F]		{Memory Write};
	\node[state]			(G) [left of=F]			{Reg Write};

	\path	(A) edge					node {} (B)
			(B) edge					node {} (C)
			(C) edge 					node {} (D)
			(D) edge 					node {} (E)
				edge 					node {} (F)
				edge 					node {} (G)
			(E) edge [bend left=100]	node {} (A)
			(F) edge [bend left=100]	node {} (A)
				edge					node {} (G)
			(G) edge 					node {} (A);

\end{tikzpicture}
\caption{Zustandsdiagramm des Steuerwerks}
\label{pic:zustandsdiagramm}
\end{figure}
\pagebreak
Die einzelnen Zustände des Steuerwerks werden im Folgenden beschrieben.
\subsubsection{Fetch}
Der zunächst auszuführende Befehl wird aus dem Hauptspeicher in den Prozessor
geladen. Die Adresse des Befehls wird durch den Wert des Programmzählers
(siehe~\ref{s:stack}) bestimmt.
\subsubsection{Decode}
In diesem Zustand wird das erste Byte des Befehl dekodiert und dabei
festgestellt, ob der Befehl noch mehr Bytes, als bisher gelesen wurden, besitzt.
Ist dies der Fall, wird wieder zum "`Fetch"'-Zustand gewechselt. Außerdem wird der
eigentlich auszuführende Befehl festgestellt, mit welchen Parametern er
auszuführen ist, aus welchen Registern die Daten kommen und wo das Resultat
gespeichert werden soll. Eine ausführlichere Beschreibung ist bei der
Beschreibung des Dekodiers in Abschnitt~\ref{s:decode} zu finden.
\subsubsection{Read}
Die Register für die Ein- und Ausgabe des Befehls werden nun ausgewählt und
dementsprechend ist genug Zeit vorhanden, die Daten zu lesen, sodass keine
korrupte Daten in den folgenden Zuständen zur Verwendung kommen.
\subsubsection{Execute}
Das Rechenwerk (siehe~\ref{s:alu}) wird bei diesem Zustand aktiviert und führt
den vorher dekodierten Befehl aus. Das Resultat des Rechenwerks wird ausgegeben,
damit in den weiteren Zuständen das Ergebnis an die gewünschte Stelle
geschrieben werden kann.
\subsubsection{Memory Write}
Das Resultat des Rechenwerks wird an die vom Befehl vorgebene Speicheradresse im
Arbeitsspeicher (siehe~\ref{s:memorycontrol}) abgelegt.
\subsubsection{Stack Read/Write}
Je nach Befehl wird entweder das Resultat des Rechenwerks auf dem Stack gelegt
oder der oberste Wert des Stacks genommen, um es im nächsten Zustand weiter
verarbeiten zu können. Eine detaillierte Erklärung der Funktionsweise des Stacks
findet in Abschnitt~\ref{s:stack} statt.
\subsubsection{Register Write}
Ähnlich wie beim "`Stack Read/Write"'-Zustand, wird auch hier (je nach Befehl) das
Ergebnis des Rechenwerks oder der Wert, der vom Stack entnommen wurde, in das
Register geschrieben, welches der Befehl vorschreibt.
\subsection{Rechenwerk}
\label{s:alu}
Das Rechenwerk ist dafür zuständig, dass die eigentliche Operation des Befehls
ausgeführt wird. Es ist zu unterscheiden, ob man von der \ac{ALU} oder vom
Rechenwerk spricht, denn die \ac{ALU} ist nur das "`Rechenzentrum"' des
Rechenwerks, gestützt von z.B. Hilfsregistern.

Der Befehlscode (auch Opcode) besteht aus 5 Bits und daher können $2^{5} = 32$
verschiedene Befehle ausgeführt werden. Mit Hilfe der
Tabelle~\ref{tab:befehlsliste} wird ein Überblick über die verfügbaren Befehle
vermittelt. Um den Aufbau simpel zu halten, wurde entschieden, dass jeder
Befehl die gleiche Bearbeitungszeit bekommt und damit jeder Befehl gleich
schnell (bezogen nur auf die Zeit beim Rechenwerk) ausgeführt wird.

Das Rechenwerk hat die Aufgabe bestimmte Statusflaggen zu setzen oder zu
löschen. Im Beispiel "`SB"' gibt die Statusflagge an, ob ein Sprungbefehl
ausgeführt wird und dieser auch die Sprungbedingung erfüllt. Der Programmzähler
erhält in diesem Fall einen neuen Wert.
\clearpage
\pagebreak
\subsection{Dekodierer}
\label{s:decode}
Der Dekodierer übernimmt die Aufgabe, den Befehl zu dekodieren, damit die
Operation für das Rechenwerk ausgewählt werden kann und die dafür benötigten
Register angesprochen werden.

Der Dekodierer ist sehr stark an den Befehlssatz angelehnt, welcher in
Abschnitt~\ref{s:befehlssatz} genauer beschrieben wird.
\subsection{Programmzähler}
\label{s:pc}
Der Programmzähler (engl. \ac{PC}) ist hardwaretechnisch gesehen ein Register,
welches aber im Vergleich zu "`normalen"' Registern ihren Wert um eine bestimmte
Byteanzahl (je nach Länge des ausgeführten Befehls) erhöht. Somit hält der \ac{PC}
immer die Adresse für den nächsten auszuführenden Befehl. Bei einem Sprungbefehl
wird der Wert des Programmzählers mit dem vom Befehl vorgeschriebenen Wert
überschrieben.

Im Ganzen lässt sich der Programmzähler als Zustandsautomat realisieren. Dazu
wurden vier verschiedene Zustände für den \ac{PC} gewählt:

\begin{labeling}{\textbf{ASSIGN}}
\item[\textbf{NOP}]		Keine Operation
\item[\textbf{INC}]		Erhöhe den Zähler um die Byteanzahl des vorherigen Befehls
\item[\textbf{ASSIGN}]	Setze den Zähler auf die Sprungadresse
\item[\textbf{RESET}]	Setze den Zähler auf null zurück
\end{labeling}

Je nach Befehl wird bei der Ausführung dann der passende Zustand ausgewählt und
somit sichergestellt, dass die Adresse für den folgenden Befehl korrekt ist.
\subsection{Stack}
\label{s:stack}
Der Stack (deutsch: Stapel) ist ein Modell zum Speichern von Daten. Dabei sind
zwei Operationen zum Speichern und Abrufen der Daten vorhanden: PUSH und POP\@.
Bei PUSH wird oben auf den Stapel ein neues Element hinzugefügt, bei POP wird
das oberste Element vom Stapel heruntergenommen. Somit ist der Zugriff auf das
oberste Element beschränkt. Der Stack wird hauptsächlich von Programmen als
Zwischenspeicher genutzt, wenn nicht genügend Register zur Verfügung stehen,
aber auch beim Aufrufen von Funktionen (Befehl CALL) wird auf dem Stack die
Rückkehraddresse gespeichert, welche beim Verlassen der Funktion (Befehl RET)
wieder vom Stapel entnommen wird.

Besonders für den Funktionsaufruf eignet sich das Prinzip des Stacks sehr gut,
da es keine Restriktionen der Anzahl von Funktionsaufrufen gibt (außer der
physikalischen Größe des Stacks). Somit ist es möglich, dass unbegrenzt viele
Funktionen ineinander aufgerufen werden können. Es sind also theoretisch
unendlich viele Verkettungen möglich, wobei die Theorie nur durch die maximale
Größe des Stacks limitiert wird.

\subsection{UART}
\label{s:uart}
Der \ac{UART}, welcher die Implementierung der seriellen Schnittstelle enthält,
wird genutzt, um Eingaben von dem Benutzer zu ermöglichen. Dabei wird über ein
selbst geschriebenes Skript die Tastatureingabe von einem anderen Rechner
übernommen und über die serielle Schnittstelle an das Gerät weitergeleitet. So
ist es möglich, eine Tastatur "`anzuschließen"'. Die Übertragung über die serielle
Schnittstelle findet mit einer Bitrate von 9600 bit/s, 8 Datenbits, 1 Stopbit
und keinem Paritätsbit statt.

Hardwaretechnisch ist der UART über die Speicheradresse 0xEF00 zu erreichen.
Sollte diese Adresse gelesen werden, wird das nächste Byte des UARTs abgefragt
und dadurch die CPU temporär gestoppt, bis ein Byte angekommen ist. Dadurch ist
die Eingabe blockierend, es können also keine weiteren Operationen während der
Eingabe durchgeführt werden.
\subsection{Speichercontroller}
\label{s:memorycontrol}
Der Arbeitsspeicher ist als kontinuierlicher Speicherblock realisiert worden. Es
sind zwei Ports für die Ein- und Ausgabe zur Verfügung, wobei der zweite Port
ausschließlich für den Videocontroller vorgesehen ist, welcher damit Zugriff auf
den Videospeicher bekommt. Dies bedeutet auch, dass der Videospeicher nicht
separat geändert werden muss und so über allgemein gültige Lese- und
Schreibbefehle angesprochen werden kann.

Insgesamt wurde der mit 16 Bit verfügbare maximal adressierbare Speicher von
$2^{16} = 65536$ Bytes ausgenutzt, wobei einzelne Abschnitte, wie beispielsweise
der Videospeicher oder auch die Speicheraddresse für den UART nicht allgemein nutzbar sind.
\subsection{Videocontroller}
\label{s:videocontrol}
Der Videocontroller selbst ist nach außen hin unabhängig von allen anderen
Elementen der \acl{EPU} (EPU). Nur der Videospeicher, welche die Informationen über die
anzuzeigenden Pixel enthält, wird vom Speichercontroller geliefert. Dennoch hat
der Speichercontroller keine Kontrolle über den Zugang des Videcontrollers zu
dem Videospeicher, denn wie bereits in~\ref{s:memorycontrol} beschrieben, ist der
Port für den Videocontroller unabhängig vom Speichercontroller.
\pagebreak
\subsection{Registerbelegung}
Die \ac{EPU} besitzt 16 Register, welche durch Selektion von $\log_2(16) = 4$
Adressbits angesprochen werden. Mithilfe der Tabelle~\ref{tab:registerbelegung}
soll eine Übersicht aller Register dargestellt werden.

\begin{table}[h]
\centering
\begin{tabular}{lll}
\toprule
Selektion & Name & Zweck\\
\midrule
0000 & R0  & Akkumulator\\
0001 & R1  & Allgemeine Verwendung\\
0010 & R2  & Allgemeine Verwendung\\
0011 & R3  & Allgemeine Verwendung\\
0100 & R4  & Allgemeine Verwendung\\
0101 & R5  & Allgemeine Verwendung\\
0110 & R6  & Allgemeine Verwendung\\
0111 & R7  & Allgemeine Verwendung\\
1000 & R8  & Allgemeine Verwendung\\
1001 & R9  & Allgemeine Verwendung\\
1010 & R10 & Allgemeine Verwendung\\
1011 & R11 & Allgemeine Verwendung\\
1100 & R12 & Allgemeine Verwendung\\
1101 & R13 & Allgemeine Verwendung\\
1110 & R14 & Flagregister\\
1111 & R15 & Programmzähler (Kopie)\\
\bottomrule
\end{tabular}
\caption{Registerbelegung}
\label{tab:registerbelegung}
\end{table}
\pagebreak
\section{Befehlssatz}
\label{s:befehlssatz}
Der Befehlssatz beschreibt den Aufbau und die Menge aller Befehle der \ac{EPU}.
Nachfolgend wird sowohl der Aufbau sowie dessen Entstehung erläutert.

\subsection{Aufbau}
Der Befehl wird byteweise gelesen. Die Anzahl an Bytes in einem Befehl sind
durch die beiden letzten Bits des ersten Bytes angegeben, dies erleichtert
die richtige Byteanzahl einzulesen. Alle weiteren Abschnitte jedes Bytes werden
je nach Opcode (die ersten 5 Bits) bestimmt, also die Register, die
Rechenoperation, ob das Ergebnis auf dem Arbeitsspeicher geschrieben werden
soll etc. Wie in Abbildung~\ref{pic:befehlsaufbau} zu sehen, ist die Angabe
von Konstanten in 4, 8 und 16 Bit möglich. Der Grund für die doch sehr
platznehmende 16-Bit-Konstante ist, dass ein Register mit einer Konstante in
einem Befehl befüllt werden soll, wobei alle Register der \ac{EPU} 16 Bit groß
sind.
\begin{figure}[htb]
\centering
\begin{tikzpicture}
	% Bitnummern oben
	\foreach \x in {0,...,7}
		\node at (\x + 0.5, 20.5) {\scriptsize \x};
	% Bitnummern unten
	\foreach \x in {0,...,7}
		\node at (\x + 0.5, 14.5) {\scriptsize \x};
	% Blaue vertikale Linien
	\foreach \x in {0,...,8}
		\draw[thick,blue] (\x, 14) -- (\x,21);
	% Das Wort ""'Bit""' nach ganz links schreiben
	\node[thick] (bit1) at (-0.5, 20.5) {\scriptsize Bit};
	\node[thick] (bit2) at (-0.5, 14.5) {\scriptsize Bit};
	% Pfeil links
	\draw[<->, thick] (-0.5, 19.9) -- (-0.5, 15.1);
	\draw[thick] (-1, 20) -- (-0.1, 20);
	\draw[thick] (-1, 15) -- (-0.1, 15);
	\node[fill=white] at (-1.1, 17.5) {\small 4 bytes};
	% Byte 0
	\filldraw[thick, draw=black, fill={rgb:white,255}] (0, 20) rectangle (5, 19);
	\node (mode) at (2.5, 19.5) {\small Opcode};
	\filldraw[thick, draw=black, fill=white] (5, 20) rectangle (6, 19);
	\node (mode) at (5.5, 19.5) {\small Flag};
	\filldraw[thick, draw=black, fill=white] (6, 20) rectangle (8, 19);
	\node (mode) at (7, 19.5) {\small Größe};
	\node[fill=white] at (9, 19.5) {\small Byte 0};
	% Byte 1
	\filldraw[thick, draw=black, fill=white] (0, 19) rectangle (4, 18);
	\node (mode) at (2, 18.5) {\small Zielregister};
	\filldraw[thick, draw=black, fill=white] (4, 19) rectangle (8, 18);
	\node (mode) at (6, 18.5) {\small Register A};
	\node[fill=white] at (9, 18.5) {\small Byte 1};
	% Byte 2
	\filldraw[thick, draw=black, fill=white] (0, 18) rectangle (4, 17);
	\node (mode) at (2, 17.5) {\small Register B};
	\filldraw[thick, draw=black, fill=white] (4, 18) rectangle (8, 17);
	\node (mode) at (6, 17.5) {\small Konstante (4 Bit)};
	\node[fill=white] at (9.8, 17.5) {\small Byte 2 Version 1};
	% Byte 2 Alternative
	\filldraw[thick, draw=black, fill=white] (0, 17) rectangle (8, 16);
	\node (mode) at (4, 16.5) {\small Konstante (8 Bit / 16 Bit [Bit 0-7])};
	\node[fill=white] at (9.8, 16.5) {\small Byte 2 Version 2};
	% Byte 3
	\filldraw[thick, draw=black, fill=white] (0, 16) rectangle (8, 15);
	\node (mode) at (4, 15.5) {\small Konstante (16 Bit [Bit 8-15])};
	\node[fill=white] at (9, 15.5) {\small Byte 3};
\end{tikzpicture}
\caption{Befehlsaufbau}
\label{pic:befehlsaufbau}
\end{figure}
\subsection{Befehlsübersicht}
Die folgende Tabelle~\ref{tab:befehlsliste} gibt einen Überblick über die
verschiedenen Befehle, welche die \ac{EPU} liefert. Durch die Größe des Opcodes (5 Bit)
sind insgesamt $2^5 = 32$ Befehle möglich. Dennoch kann ein Befehl durch
bestimmte Parameter (wie z.B. der Flag im ersten Byte) um gewisse Funktionen
erweitert werden. Dies ermöglicht beispielsweise vorzeichenbehaftete und
vorzeichenlose Additionen, obwohl der Opcode beider Operationen der Gleiche ist.
Eine ausführlichere Beschreibung jedes Befehls ist im
Anhang~\ref{a:befehlsliste} zu finden.
\begin{table}[htb]
\centering
\begin{tabular}{lll}
\toprule
Beschreibung									& Kurzform	& Opcode\\
\midrule
Keine Operation									& NOP		& 00000\\
Addition										& ADD   	& 00001\\
Subtraktion										& SUB   	& 00010\\
Logisches UND									& AND   	& 00011\\
Logisches ODER									& OR    	& 00100\\
Logisches XOR									& XOR   	& 00101\\
Logisches NOT									& NOT   	& 00110\\
Laden eines Registers							& LOAD  	& 00111\\
Verschieben eines Registerwertes				& MOV   	& 01000\\
Lesevorgang im Arbeitsspeicher					& READ  	& 01001\\
Schreibvorgang im Arbeitsspeicher				& WRITE 	& 01010\\
Bitweises Verschieben nach links				& SHL   	& 01011\\
Bitweises Verschieben nach rechts				& SHR   	& 01100\\
Vergleichen zweier Register						& CMP   	& 01101\\
Springen zu einer Adresse						& JMP   	& 01110\\
Bedingtes Springen zu einer Adresse				& JC    	& 01111\\
Nicht benutzt									& ---   	& 10000\\
Aufrufen einer Funktion							& CALL  	& 10001\\
Zurückkehren von einer Funktion					& RET   	& 10010\\
Ein Element auf den Stack schieben				& PUSH  	& 10011\\
Von dem Stack ein Element entnehmen				& POP   	& 10100\\
Nicht benutzt									& ---   	& 10101\\
Flags für die Beziehung zweier Register setzen	& TEST  	& 10110\\
Nicht benutzt									& ---   	& 10111\\
Nicht benutzt									& ---   	& 11000\\
Stoppen der Ausführung des Prozessors			& HLT   	& 11001\\
Setzen eines bestimmten Bits eines Registers	& SET   	& 11010\\
Löschen eines bestimmten Bits eines Registers	& CLR   	& 11011\\
Nicht benutzt									& ---   	& 11100\\
Nicht benutzt									& ---   	& 11101\\
Nicht benutzt									& ---   	& 11110\\
Nicht benutzt									& ---   	& 11111\\
\bottomrule
\end{tabular}
\caption{Befehlsliste}
\label{tab:befehlsliste}
\end{table}

\chapter{Implementierung in VHDL}
\label{c:vhdl}
\section{Einführung in VHDL}
In diesem Abschnitt soll eine kurze Einführung in VHDL gegeben werden, damit die
nachfolgende Erklärung der Implementierung leichter zu verstehen ist.

\subsection{Was ist VHDL}
VHDL ist eine sogenannte Hardware Description Language (HDL), mit welcher es möglich ist, digitale
Schaltungen in Form von Quellcode zu schreiben, um diesen dann von einem
Computer zu Hardware weiterverarbeiten zu lassen. VHDL ist eine Abkürzung für
VHSIC HDL (\acl{VHDL}). Der geschriebene VHDL-Code wird dann synthesiert zu
einer Netzliste, welche die einzelnen Verbindungen zwischen den Komponenten
beschreibt. Diese Netzliste wird daraufhin für den jeweiligen \ac{FPGA}
übersetzt, sodass als letzter Schritt eine Konfigurationsdatei erstellt werden
kann, welche auf den \ac{FPGA} geladen werden kann.

Eine Besonderheit bei VHDL ist, dass nicht jeder geschriebene Quellcode
synthesierbar ist, was bedeutet, dass mancher Quellcode nur in der Simulation
funktioniert. In dem Umfang dieses Projektes ist aber der geschriebene Quellcode
vollkommen synthesierbar und soll nicht nur in der Simulation funktionieren.

\subsection{Notation}
Allgemein endet jede Anweisung immer mit einem Semikolon. Kommentare werden mit
einem doppelten Bindestrich (-{}-) eingeleitet und sind gültig bis zum Ende der
Zeile. Einzelne Bitwerte werden mit einfachen Anführungszeichen und Werte
von Bitvektoren (mehreren Bits) mit doppelten Anführungszeichen umrahmt.

Außerdem gibt es noch mehrere Sprachelemente, welche häufig Verwendung
haben. Eine (nicht vollständige) Beschreibung von häufig genutzten
Sprachelementen:

\begin{description}[align=right, labelwidth=1.6cm]
\item[signal]		Verbindung zweier Module. Beschreibt die Leitung zwischen den
					Modulen.
\item[variable]		Ein Zwischenspeicher für Werte, welcher \textbf{nicht} als
					Signal übersetzt wird.
\item[entity]		Schlüsselwort zur Deklaration eines Moduls (logische Einheit).
\item[in]			Deklaration eines Eingabesignals eines Moduls.
\item[out]			Deklaration eines Ausgabesignals eines Moduls.
\item[architecture]	Schlüsselwort zur Definition eines Moduls
\item[component]	Signatur eines Moduls. Wird verwendet, um ein Modul
					innerhalb eines anderen Moduls zu verwenden.
\item[package]		Eine Sammlung (Bibliothek) von \textbf{components}.
\end{description}
\pagebreak
\section{Beschreibung wichtiger Module}
\subsection{top -- Verbindung der Ein- und Ausgänge}
\label{s:top}
Das Topmodul dient als oberste Ebene, welche die \ac{EPU} mit der `Außenwelt'
verbindet. Die dazugehörigen Ein- und Ausgangssignale werden in der Datei
\textit{top.vhdl} festgelegt.

Wie in Abbildung~\ref{code:top} zu erkennen, wird zuerst der Takt in Zeile drei als
Eingang definiert, in den Zeilen vier bis acht die 7-Segment-Anzeige, die LEDs
und Taster definiert. Die Zeilen zehn und elf beschreiben die beiden Signale des
seriellen Schnittstelle und Zeilen dreizehn bis siebzehn beschreiben die Signale
der VGA--Verbindung.
\begin{figure}[htb]
\lstinputlisting[firstline=11, lastline=29]{../vhdl/top/top.vhdl}
\caption{Topmodul}
\label{code:top}
\end{figure}
\pagebreak
\subsection{core -- Topmodul der CPU}
\label{s:core}
Das `core'-Modul verbindet die einzelnen Baugruppen der CPU\@. Dazu gehören
Rechenwerk, Steuerwerk, Dekodierer, Programmzähler und der Stack, welche bereits
in Kapitel~\ref{c:hardware} beschrieben wurden. 

In Abbildung~\ref{code:core} ist die Definition des Moduls dargestellt. Die
Signalnamen wurden dem Zweck angepasst, dass heißt das Verbindungen zum
Speichercontroller mit `MEM' und interne Signale mit `CORE' bezeichnet. Das
Signal `O\_LED' ist dabei eine Ausnahme, da dieser zu Testzwecken verwendet wird
und damit nicht fest eingeordnet werden kann.
\begin{figure}[htb]
\lstinputlisting[firstline=11, lastline=31]{../vhdl/core/core.vhdl}
\caption{`core'-Modul}
\label{code:core}
\end{figure}
\pagebreak
\subsection{memory\_control -- Speichercontroller}
\label{s:memcontrol}
Der Speichercontroller wurde bereits in Abschnitt~\ref{s:memorycontrol}
beschrieben und hier soll nur die Implementation dessen dargestellt werden.

Wie auch bei dem `core'-Modul (siehe~\ref{s:core}) sind hier die Signale nach
Zugehörigkeit bezeichnet und interne Signale mit `MEM' und Videosignale mit
`VID' beschrieben. Außerdem sind die letzten drei Signale (Zeilen 22--24) für
die serielle Schnittstelle vorgesehen, welches im Speicher abgebildet wird.
\begin{figure}[htb]
\lstinputlisting[firstline=11, lastline=36]{../vhdl/mem/memory_control.vhdl}
\caption{Speichercontroller}
\label{code:mem_control}
\end{figure}
\pagebreak
\subsection{alu -- Rechenwerk}
Das Rechenwerk führt die eigentliche logische bzw. mathematische Operation aus.
Hier soll zum Verständnis nur ein Teil des Quellcodes gezeigt werden.

Die nachfolgende Abbildung~\ref{code:alu_add} beschreibt den Ablauf der \ac{ALU} bei dem Befehl
'ADD'.

\begin{figure}[htb]
\lstinputlisting[firstline=65, lastline=89]{../vhdl/core/alu.vhdl}
\caption{Addition aus Sicht der ALU}
\label{code:alu_add}
\end{figure}
Der Quellcode ist recht einfach zu lesen, denn die eigentliche Logik, die hier
passiert, basiert auf Abfragen nach bestimmten Flags. So gibt es die Möglichkeit
beim 'ADD'-Befehl einerseits vorzeichenbehaftet und andererseits
vorzeichenunbehaftet zu addieren (siehe Abfrage Zeile 2). Danach wird noch bei
beiden Fällen noch überprüft, ob zwei Register oder ein Register und eine
Konstante addiert werden sollen (siehe Zeilen 3 und 13). Die je nach Fall
ausgeführte Addition ist dann trivial, da sie bereits in die Sprache eingebunden
ist und nur die Werte in die richtigen Datentypen (\textit{signed} \&
\textit{unsigned}) konvertiert werden müssen (siehe Zeilen 5-6, 9-10, 15 und
20).  Bei der vorzeichenbehafteten Addition muss die Erkennung eines
\textbf{Überlaufs} seperat durchgeführt werden (siehe Zeilen 16-17 und 21-22),
wohingegen bei der vorzeichenunbehafteten Addition die Erkennung einfach am
siebzehnten Bit zu erkennen ist und damit innerhalb der eigentlichen Rechnung
stattfindet.

\captionsetup[figure]{justification=justified,singlelinecheck=false}
\chapter{Programmstruktur}
In diesem Kapitel soll eine Übersicht über die Struktur und den Aufbau der
Programme, welche die Software der EPU bilden, dargestellt werden. Die
Voraussetzung für das Verständnis der Umsetzung der Software in die Hardware
ist bereits in Kapitel~\ref{c:hardware} (oder genauer bei der Erklärung des
Befehlssatzes in Abschnitt~\ref{s:befehlssatz}) geschehen.
\section{Einführung in EPU-Assembly}
Zur Übersetzung der eigenen Assembly-Sprache in Maschinencode, welche die CPU
versteht, wurde ein eigener Assembler entwickelt. Der Assembler ist in der
Interpretersprache Python geschrieben und erlaubt die Kompilierung von einer
.easm-Datei, wobei eine Vorlage mit vordefinierten Funktionen als externe Datei
auch noch übergeben werden kann. Die Ausgabe erfolgt durch das Schreiben der
einzelnen Bytes des Maschinencodes in eine Datei im .coe-Format. Der Grund für
das .coe-Format ist, dass dieses genutzt werden kann, um den RAM-Block der EPU
zu initialisieren.

Der Aufbau der Befehle, welche der Assembler versteht, wurde einfach
gehalten, um eine leicht zu verstehende Sprache zu erstellen, wobei viele
Ähnlichkeiten zur x86-Assembly (genutzt in den meisten Desktopcomputern
heutzutage) bestehen, da somit das anfängliche Verstehen der Befehle leichter
wird. Der Aufbau eines Befehls richtet sich nach folgender Regel, wobei in
eckigen Klammern eingerahmte Parameter optional sind bzw.\ nur bei bestimmten
Befehlen notwendig sind:
\begin{lstlisting}[basicstyle=,numbers=none]
	Mnemonic[.Option] [Operand1] [,Operand2] [,Operand3]
\end{lstlisting}

Um einen Befehl aufzurufen, wird das obere Schema verwendet, was bedeutet, dass
der Befehl selbst mit seiner \textbf{Mnemonic} startet. Das Wort Mnemonic
bedeutet auf Deutsch so viel wie "`Merkspruch"'. Es wird meist bei
Assemblersprachen verwendet, um beim Programmieren das Auswendiglernen des
Maschinencodes jedes Befehls zu sparen. An die Mnemonic kann eine Option nach
einem Punkt angehängt werden. Dies ist befehlsspezifisch und muss für jeden
Befehl nachgeschlagen werden. Je nach Befehl werden auch ein bis zwei weitere
Operanden benötigt, welche selbst noch in verschiedenen Formen vorkommen können
(abhängig von der gewählten Option). Um dies zu verdeutlichen, sollen ein
paar Beispiele für den Befehlsaufbau gezeigt werden.
\begin{figure}
	\begin{lstlisting}[basicstyle=]
	load r1, 0xAFFE
	jmp.i $test

	data:
		.data 0xFF

	test:
		addi r2, r1, 0d10
		load r1, $data
		write.l r1, r2, 0x0
\end{lstlisting}
\caption{Befehlsaufbau - Assembly}
\label{code:instruction_code}
\end{figure}

In Abbildung~\ref{code:instruction_code} wird zuerst in Zeile 1 eine
hexadezimale Konstante (Präfix "`0x"' steht für hexadezimal) in das Register R1
geladen. Danach wird ein Sprungbefehl mit der Option "`i"' ausgeführt, wobei das
"`i"' für "`Immediate"' bzw. Konstante steht und damit die Adresse, zu welcher
gesprungen werden soll, im Befehl vorhanden ist.  Die Konstant ist hier mit dem
Adressoperator "`\$"' vorangestellt, welcher die Adresse des \textbf{Labels}, hier
"`test"' (Zeile 7), enthält. Dadurch wird die Ausführung in Zeile 8, also direkt
nach der Definition des Labels, fortgeführt.

Davor ist aber noch (in Zeile 5) eine sogennante \textbf{Assemblerdirektive}
zu sehen, welche an dem "`."' vor der Mnemonic zu erkennen ist.
Assemblerdirektiven sind Anweisungen an den Assembler und werden nicht direkt in
Maschinencode umgewandelt, sondern getrennt behandelt. In diesem Beispiel ist
die Assemblerdirektive "`data"' verwendet worden, welche die angegebene Konstante
im ersten Operanden unmittelbar in die Byteausgabe des Maschinencodes übernimmt.
So ist es beispielsweise möglich, dass vordefinierten Variablen bereits einen
Wert beim Start des Programms zugewiesen bekommen können.

Zurück bei der Ausführung des Beispiels in Zeile 8: Hier wird mit dem Befehl
"`addi"' eine Konstante zu dem Register R1 (zweiter Operand) eine dezimale
Konstante (Präfix "`0d"' für dezimale Konstanten) addiert. Das Ergebnis wird im
ersten Operand gespeichert - hier R2. In Zeile 9 wird die Adresse des
"`data"'-Labels in das Register R1 geladen und danach in Zeile 10 wird an die
Speicheradresse in R1 der Wert im Register R2 geschrieben. Der dritte Operand
wird hier nicht benötigt und daher null gesetzt, kann aber für einen Offset der
Speicheradresse dienen.
\pagebreak
\section{Funktionsaufruf}
In diesem Abschnitt soll der Aufruf von Funktionen demonstriert werden. Bei dem
Aufbau wurde eine ähnliche Methode wie beim x86-Befehlssatz verwendet. Die
einfachste Methode des Funktionsaufrufs ist das Aufrufen einer Funktion ohne
Parameter, wie unten in Abbildung~\ref{code:function_struct}  zu sehen ist.

\begin{figure}[htb]
\begin{lstlisting}[basicstyle=]
funktion:
	ret

_start:
	call.i $funktion
\end{lstlisting}
\caption{Funktionsaufruf ohne Parameter}
\label{code:function_struct}
\end{figure}

Dabei wird einfach der Befehl "`call"' aufgerufen mit dem ersten Operanden hier
als Adresse zum Label "`funktion"'. Nachdem die Funktion mit dem Befehl "`ret"'
wieder zurückkehrt, wird automatisch der nächste Befehl nach dem Funktonsaufruf
ausgeführt.

Damit beim Zurückkehren von einer Funktion die Rücksprungadresse für den
nächsten Befehl gefunden werden kann, wird beim Aufruf von "`call"' nicht nur ein
Sprungbefehl zur Funktion aufgerufen, sondern auch die Rückkehradresse auf den
Stack (siehe~\ref{s:stack}) gelegt. Beim Ausführen von "`ret"' wird dann einfach
das Element vom Stack heruntergenommen und in den PC (siehe~\ref{s:pc})
geschrieben.

Bei Funktionen mit Parametern und Rückgabewert ist der Aufbau ähnlich. Das
folgende Beispiel in Abbildung~\ref{code:function_parameter} soll einen solchen
Funktionsaufruf illustrieren.

Zuerst werden die zu übergebenen Argumente in beliebige Register gespeichert
(Zeile 15-16) und dann in \textbf{umgekehrter} Reihenfolge auf den Stack gelegt
(Zeile 17-18). Anschließend findet der eigentliche Funktionsaufruf statt und die
Funktion wird aufgerufen, wobei unter anderem die Rückkehradresse auf dem Stack
abgelegt wird. Da die Rückkehradresse ganz oben auf dem Stack abgelegt ist, muss
diese erst vom Stack heruntergenommen und in einem Register zwischengespeichert
werden, damit die Argumente vom Stack geholt werden können (Zeile 2). Da sie in
umgekehrter Reihenfolge auf den Stack gelegt wurden, dreht sich beim Holen der
Argumente die Reihenfolge wieder um, sodass die Reihenfolge wieder die
ursprüngliche ist (Zeile 3-4). Danach kann die Rückkehradresse wieder zurück auf
den Stack gelegt werden (Zeile 5). Nun kann die eigentliche Operation der
Funktion durchgeführt werden; dies ist hier die Subtraktion (Zeile 7). Da nun
der Rückgabewert auf dem Stack abgelegt werden soll, aber auch zugriffsbereit
für die aufrufende Funktion, muss zuerst wieder die Rückkehradresse vom Stack
entfernt (Zeile 9) und dann der Rückgabewert auf den Stack gelegt werden (Zeile
10). Jetzt kann die Rückkehradresse wieder auf den Stack (Zeile 11) und die
Funktion kann wie gewohnt zurückkehren (Zeile 12).

Durch die Nutzung des Stacks für den Rückgabewert ist es theoretisch möglich,
mehrere Rückgabewert einer Funktion zuzuweisen. In dem Beispiel ist dies aber
nicht der Fall. Dennoch ist zu erwähnen, dass über die Register auch ein oder
mehrere Rückgabewerte theoretisch erfolgen können. Genauere Informationen sind
der jeweiligen Funktion, welche man aufrufen möchte, zu entnehmen.
\begin{figure}[!htb]
\begin{lstlisting}[basicstyle=]
sub:
	pop r15
	pop r0
	pop r1
	push r15

	sub.u r0, r0, r1

	pop r15
	push r0
	push r15
	ret

_start:
	load r4, 0x0005
	load r5, 0x0002
	push r5
	push r4
	call.i $sub
	pop r3
\end{lstlisting}
\caption{Funktionsaufruf mit Parametern}
\label{code:function_parameter}
\end{figure}

\clearpage
\section{Vordefinierte Funktionen}
Einige hilfreiche Funktionen sind bereits vorgeschrieben worden und
können als Vorlage an den Assembler übergeben werden. Dadurch können diese
vordefinierten Funktionen in allen Programmen genutzt werden. Unter anderem sind
folgende Funktionen verfügbar:
\begin{labeling}{\textbf{\_print(*data, start)}}
\item [\textbf{\_print(*data, start)}] Gebe eine null-terminierte Zeichenkette an der
Stelle \textbf{data} auf dem Bildschirm an Startpositon \textbf{start}
aus.
\item [\textbf{\_setcursor(pos)}] Setze den Bildschirmcursor an Position \textbf{pos},
	wobei der Wert 0xAABB die x-Koordinate 0xAA und die y-Koordinate 0xBB setzt.
\item [\textbf{\_getinput}] Liest ein Byte vom Benutzer ein und gibt es zurück.
\item [\textbf{\_getstring}] Liest einen "`Enter"'-terminierten String vom Benutzer ein und
	gibt diesen zurück
\item [\textbf{\_set\_vga}] Ermöglicht es, die Optionen des VGA-Ausgangs zu verändern, um
	z.B. andere Farben zu verwenden oder den VGA-Ausgang zu deaktivieren.
\end{labeling}
Die Benutzung dieser Vorlage wird empfohlen, ist aber keine Voraussetzung. Denn
die Vorlage definiert auch, dass der Start des eigentlichen Programms beim Label
"`\_start"' stattfindet und dieser auch von der Vorlage aufgerufen wird. Wie
bereits aufgefallen ist, starten bei der Erstellung der Vorlage
alle genutzten Labels mit einem Unterstrich, was den Grund hat, alle
vordefinierten Funktionen und die Funktionen des Programms visuell zu trennen.
Daher ist es auch nicht empfohlen, dass im Programm Labels mit einem Unterstrich
starten.
\pagebreak
\section{Debugging}
Mithilfe des Debugging ist es möglich, Fehler in Programmen zu erkennen und
diese auch zu beseitigen. Damit dies auch bei der \ac{EPU} möglich ist, soll
dieser Abschnitt beispielhaft anhand eines Signal-Zeit-Diagrammes zeigen, wie die
Ausführung eines Befehls aussieht.
Der auszuführende Befehl sei folgender, wobei angenommen wird das Register R14
bereits den Wert 0x1234 hat:
\begin{lstlisting}[basicstyle=,numbers=none]
	push r14
\end{lstlisting}

\captionsetup[figure]{justification=centering,singlelinecheck=false}
\begin{figure}[htb]
\raggedright
\begin{wave}{11}{9}{0}
 \nextwave{en\_fetch} \bit{1}{1} \bit{0}{1} \bit{1}{1} \bit{0}{1} \bit{1}{1}
					 \bit{0}{5}
 \nextwave{en\_decode} \bit{0}{1} \bit{1}{1} \bit{0}{1} \bit{1}{1} \bit{0}{1}
					   \bit{1}{1} \bit{0}{4}
 \nextwave{en\_regread} \bit{0}{6} \bit{1}{1} \bit{0}{3}
 \nextwave{en\_alu} \bit{0}{7} \bit{1}{1} \bit{0}{2}
 \nextwave{en\_stack} \bit{0}{8} \bit{1}{1} \bit{0}{1}
 \nextwave{en\_regwrite} \bit{0}{9} \bit{1}{1}
 \nextwave{pc} \known{0x0}{2} \known{0x1}{2} \known{0x2}{6}
 \nextwave{dataA} \unknown{7} \known{0x1234}{3}
 \nextwave{stack\_i\_en} \bit{0}{8} \bit{1}{1} \bit{0}{1}
 \nextwave{stack\_i\_we} \bit{0}{8} \bit{1}{1} \bit{0}{1}
\end{wave}
\caption{Signal-Zeit-Diagramm}
\label{wave:program}
\end{figure}

Wie in Abbildung~\ref{wave:program} zu sehen ist, beschreiben die ersten 6
Signale die einzelnen Zustände des Steuerwerks (siehe~\ref{s:control}), danach
folgt der Programmzähler (siehe~\ref{s:pc}) und der Wert des übergebenen
Registers Ra, gefolgt von dem "`Enable"'- und "`WriteEnable"'-Eingang des Stacks
(siehe~\ref{s:stack}).

Im ersten Abschnitt von t=0 bis t=6 ist zu erkennen, wie byteweise der Befehl
gelesen und dekodiert wird. Dabei wird bei jeder positiven Flanke des Signals
"`en\_fetch"' der Programmzähler inkrementiert, damit er auf das nächste zu
dekodierende Byte zeigt. Im zweiten Abschnitt von t=6 bis t=9 werden zuerst die
Werte der übergebenen Register gelesen. Diese sind dann nach der negativen
Flanke von "`en\_regread"' verfügbar (siehe Signal "`dataA"'). Als nächstes wird das
Rechenwerk aktiviert und legt in diesem Fall fest, dass kein Sprungbefehl
ausgeführt werden muss. Danach wird "`en\_stack"' auf 1 gesetzt (t=8) und damit
auch die "`Enable"'- und "`WriteEnable"'-Eingänge des Stacks aktiv. Der Stack
übernimmt daraufhin den Wert von "`dataA"' und schreibt ihn als oberstes Element
auf den Stapel. Zuletzt wird "`en\_regwrite"' aktiv, da aber dies ohne Bedeutung
bei diesem Befehl ist, passiert hier nichts.

\chapter{Fazit}
\label{c:fazit}
\section{Umfang und Aufwand}
Der Aufwand der besonderen Lernleistung war insgesamt deutlich höher als
anfänglich gedacht. Der Einstieg in das Thema ging ziemlich schnell, sodass die
Simulation des ersten Programm bereits ein Monat nach Beginn funktionierte.

Nachfolgend gab es kleinere Probleme, bei denen die Behebung durch die
Simulation deutlich vereinfacht wurde. Als es dann zur ersten Übertragung des
Projektes auf den \ac{FPGA} kam, trat ein größeres Problem auf. Zuerst konnte die
Datei für den \ac{FPGA} nicht übertagen werden, da der Hersteller nur ein Programm
für Windows zur Übertragung anbietete, aber der Rest des Projektes bereits
vollständig auf Linux erstellt wurde. Nach etwas Recherche stellte sich dann
aber raus, dass jemand bereits das selbe Problem für ein ähnliches Board des
selben Herstellers hatte und seinen Code für ein Programm zur Übertragung
veröffentlicht hatte~\cite{numatoloader}. Dies war eine große Hilfe, da nun nur noch die
Beschreibung des Boards im Code angepasst werden musste und somit auch die
Übertagung funktionierte.

Wenig später sollte auch der RAM-Block des \ac{FPGA} genutzt werden, wobei
vorher nur für die Simulation der Speicher aus Flip-Flops erstellt wurde. Dabei
stellte sic das Problem, dass die Dokumentation des DDR-RAM-Chips auf dem Board
nicht sehr ausführlich ist und keine große Hilfe bei der Implementierung bietet.
Deshalb wurde kein DDR-RAM verwendet, sondern der interne Speicher des
\ac{FPGA}-Chip. Da nur 64 Kilobyte an Speicher für die 16-Bit-Adressierung der
\ac{EPU} nötig sind, reicht dieser interne Speicher vollkommen aus und
vereinfacht damit auch den Aufbau der \ac{EPU}, da kein komplexer
DDR-Speichercontroller implementiert werden musste.

Auch wurde anfangs überlegt, alle Grundrechenarten als eigene Befehle zu
implementieren. Nach langer Überlegung wurde aber dagegen entschieden und nur
die Addition und Subtraktion in die Hardware implementiert, da die anderen
Rechenarten über die Software später implementiert werden können und nicht
essentiell sind. Grundsätzlich lohnte sich nicht der notwendige Zeitaufwand der
Implementierung für die Multiplikation und Division, da diese Befehle nicht
häufig genug Verwendung bekommen.

\section{Ziele und Lernerfolg}
Hauptziel der besonderen Lernleistung war es, einen funktionsfähigen
Mikrorechner bzw. zu produzieren und dieser Ziel ist auch sehr gut gelungen.
Außerdem sollte eine Ausgabe über einen Bildschirm als auch eine Eingabe über
eine Tastatur möglich sein, was beides funktioniert.

Andere anfangs sehr übernommene Ziele, wie eine Netzwerkverbindung, Soundtreiber
oder externer Speicher über den MicroSD-Kartenslot, mussten aus Zeitgründen
leider gestrichen werden. Dennoch sind diese Ziele als mögliche Erweiterungen zu
sehen, welche im Nachhinein noch implementiert werden könnten.

Der Lernerfolg bei diesem Projekt war außerordentlich hoch. Nicht nur die
Sprache VHDL und deren Syntax wurde gelernt und angewendet, sondern viel mehr
wurde über den Aufbau eines Rechnes, insbesondere eines Prozessors, gelernt.
Auch wie Hardware entwickelt wird und worauf dabei zu achten ist, darf nicht
unterschätzt werden. Diese Arbeit auf einer sehr niedrigen Ebene ('low-level')
hatte sehr viele Vorteile und verbesserte das allgemeine Verständnis, wie ein
Rechner funktioniert und hilft auch bei anderen Projekten, die nicht so nah an
der Hardware liegen. Besonders beim Beheben von Fehlern in Software ist dieses
Wissen nicht ersetzbar und stellt dabei eine große Hilfe dar.
\pagebreak


% Literaturverzeichnis
\bibliography{literatur}

\end{document}
