% Dokumentenklasse
\documentclass[a4paper,12pt,liststotoc, parskip=half]{scrreprt}

% ============ Pakete ============
% Dokumentinformationen
\usepackage[
	pdftitle={Dokumentation EPU},	
	pdfsubject={},
	pdfauthor={Markus Schneider},
	pdfkeywords={},
	% Links nicht einrahmen
	hidelinks
]{hyperref}

% Standardpakete
\usepackage[utf8]{inputenc}				% UTF-8 Zeichensatz
\usepackage[ngerman]{babel}				% Alle Bezeichnungen auf die deutsche Sprache anpassen
\usepackage[T1]{fontenc}				% Unterstützung für westeuropäische Codierung(Umlaute)
\usepackage{graphicx}					% Grafiken einbinden
\usepackage{subfig}						% Abbildungen und Tabellen
\graphicspath{{img/}}					% Pfad zu Grafiken
\usepackage{fancyhdr}					% Einfache Bearbeitung von Kopf- und Fußzeile
\usepackage{lmodern}					% Verändert die Schriftart auf "Latin Modern"
\usepackage{color}						% Farbenmanagement
\usepackage[printonlyused]{acronym}		% Abkürzungsverzeichnis
\usepackage{booktabs}					% Tabellen ("Publication Quality")
\usepackage{setspace}					% Abstände
\usepackage{pgf}						% Makropaket zum Erstellen von Grafiken
\usepackage{tikz}						% Vektorgrafiksprache
\usepackage{bm}							% Fettschrift für mathematische Objekte
\usepackage{mathptmx}					% Adobe Times Roman als Schriftart
\usepackage{float}						% Verbessert die Schnittstelle für Gleitobjekte
\usepackage{enumitem}					% Kontrolle über das Layout von Auflistungen
\usepackage{listingsutf8}				% Quellcode
\usepackage{todonotes}					% TODO-Liste

% Quellcodeumlaute
\lstset{literate=
  {á}{{\'a}}1 {é}{{\'e}}1 {í}{{\'i}}1 {ó}{{\'o}}1 {ú}{{\'u}}1
  {Á}{{\'A}}1 {É}{{\'E}}1 {Í}{{\'I}}1 {Ó}{{\'O}}1 {Ú}{{\'U}}1
  {à}{{\`a}}1 {è}{{\`e}}1 {ì}{{\`i}}1 {ò}{{\`o}}1 {ù}{{\`u}}1
  {À}{{\`A}}1 {È}{{\'E}}1 {Ì}{{\`I}}1 {Ò}{{\`O}}1 {Ù}{{\`U}}1
  {ä}{{\"a}}1 {ë}{{\"e}}1 {ï}{{\"i}}1 {ö}{{\"o}}1 {ü}{{\"u}}1
  {Ä}{{\"A}}1 {Ë}{{\"E}}1 {Ï}{{\"I}}1 {Ö}{{\"O}}1 {Ü}{{\"U}}1
  {â}{{\^a}}1 {ê}{{\^e}}1 {î}{{\^i}}1 {ô}{{\^o}}1 {û}{{\^u}}1
  {Â}{{\^A}}1 {Ê}{{\^E}}1 {Î}{{\^I}}1 {Ô}{{\^O}}1 {Û}{{\^U}}1
  {œ}{{\oe}}1 {Œ}{{\OE}}1 {æ}{{\ae}}1 {Æ}{{\AE}}1 {ß}{{\ss}}1
  {ű}{{\H{u}}}1 {Ű}{{\H{U}}}1 {ő}{{\H{o}}}1 {Ő}{{\H{O}}}1
  {ç}{{\c c}}1 {Ç}{{\c C}}1 {ø}{{\o}}1 {å}{{\r a}}1 {Å}{{\r A}}1
  {€}{{\euro}}1 {£}{{\pounds}}1 {«}{{\guillemotleft}}1
  {»}{{\guillemotright}}1 {ñ}{{\~n}}1 {Ñ}{{\~N}}1 {¿}{{?`}}1
}

% Quellcodefarben
\definecolor{codegreen}{rgb}{0,0.6,0}
\definecolor{codegray}{rgb}{0.5,0.5,0.5}
\definecolor{codemauve}{rgb}{0.58,0,0.82}

% Quellcodeoptionen
\lstset{
	basicstyle=\footnotesize,
  	breakatwhitespace=false,
  	breaklines=true,
  	captionpos=b,
  	commentstyle=\color{codegreen},
  	extendedchars=true,
  	frame=none,
  	keepspaces=true,
  	keywordstyle=\color{blue},
	language=vhdl,
  	numbers=left,
  	numbersep=5pt,
  	numberstyle=\color{codegray},
  	rulecolor=\color{black},
  	showspaces=false,
  	showstringspaces=false,
  	showtabs=false,
	stepnumber=1,
  	stringstyle=\color{codemauve},
	tabsize=4,
	% title=\lstname,
	backgroundcolor=\color{white}
}

% Vektorgrafikbibliotheken
\usetikzlibrary{arrows,automata,decorations.pathmorphing,backgrounds,fit,positioning,shapes.symbols,chains,shapes.geometric,shapes.arrows,calc}

% Zusätzliche Schriftzeichen der American Mathematical Society
\usepackage{amsfonts}
\usepackage{amsmath}

% Zusätzliche Befehle
\newcommand{\mx}[1]{\mathbf{\bm{#1}}} % Matrix command
\newcommand{\vc}[1]{\mathbf{\bm{#1}}} % Vector command
\newcommand{\titledate}[2][2.5in]{%
  \noindent%
  \begin{tabular}{@{}p{#1}@{}}
    \\ \hline \\[-.75\normalbaselineskip]
    #2
  \end{tabular} \hspace{1in}
  \begin{tabular}{@{}p{#1}@{}}
    \\ \hline \\[-.75\normalbaselineskip]
    Datum
  \end{tabular}
}

% ============ Kopf- und Fußzeile ============
% Style
\pagestyle{fancy}

% Kopfzeile
\lhead{}
\chead{}
\rhead{\slshape \leftmark}

% Fußzeile
\lfoot{}
\cfoot{\thepage}
\rfoot{}

% Trennlinien
\renewcommand{\headrulewidth}{0.4pt}	% Dünne Trennline nach der Kopfzeile
\renewcommand{\footrulewidth}{0pt}		% Keine Trennline vor der Fußzeile

% ============ Paketeinstellungen & Sonstiges ============
% Besondere Trennungen
\hyphenation{De-zi-mal-tren-nung}

% ============ Dokumentbeginn ============
\begin{document}

% Seiten ohne Kopf- und Fußzeilen sowie Seitenzahl
\pagestyle{empty}

% Festlegung der Art der Zitierung
\bibliographystyle{unsrt}

\title{Implementierung eines Mikrorechners in VHDL auf einem FPGA}
\date{\today}
\author{Markus Schneider}
% ----------------------------------------------------------------------------------------------------------
% Titelseite
% ----------------------------------------------------------------------------------------------------------
\thispagestyle{empty}
\begin{center}
	\includegraphics[scale=0.33]{img/bsgg.png}\\
	\vspace*{2cm}
	\Large
	\textbf{Besondere Lernleistung}\\
	\textbf{Berufliche Schulen Groß-Gerau}\\
	\vspace*{2cm}
	\Huge
	\textbf{Implementierung eines Mikrorechners in VHDL auf einem FPGA}\\
	\vspace*{0.5cm}
	\Large
	% über das Thema\\
	\vspace*{1cm}
	\textbf{Technikwissenschaft \& Technologie}\\
	\vspace*{2cm}
	
	\vfill
	\large
	\newcolumntype{x}[1]{>{\raggedleft\arraybackslash\hspace{0pt}}p{#1}}
	\begin{tabular}{x{6cm}p{7.5cm}}
		\rule{0mm}{4ex}\textbf{Autor:} & Markus Schneider \\
		\rule{0mm}{4ex}\textbf{Klasse:} & 13BGDE \\
		\rule{0mm}{4ex}\textbf{Wohnort:} & Mörfelden-Walldorf \\
		\rule{0mm}{4ex}\textbf{Prüfer:} & Uwe Homm, Friedrich Ernst \\ 
		\rule{0mm}{4ex}\textbf{Abgabedatum:} & 30.03.2017 \\ 
	\end{tabular} 
\end{center}
\pagebreak


% Beendet eine Seite und erzwingt auf den nachfolgenden Seiten die Ausgabe aller Gleitobjekte
% (z.B. Abbildungen), die bislang definiert, aber noch nicht ausgegeben wurden. Dieser Befehl fügt,
% falls nötig, eine leere Seite ein, sodaß die nächste Seite nach den Gleitobjekten eine ungerade
% Seitennummer hat. 
\cleardoubleoddpage{}

% Seiten ab jetzt mit Kopf- und Fußzeilen sowie Seitenzahl
\pagestyle{fancy}

% Inhaltsverzeichnis
\tableofcontents

% Abkürzungsverzeichnis
\begin{acronym}

\thispagestyle{empty}
\chapter*{Abkürzungsverzeichnis}
\addcontentsline{toc}{chapter}{Abkürzungsverzeichnis}
\acro{EPU}{Educational Processing Unit}
\acro{CPU}{Central Processing Unit}
\acro{FPGA}{Field-programmable gate array}
\acro{IC}{Integrated Circuit}
\acro{LUT}{Look-Up-Table}
\acrodefplural{LUT}[LUTs]{Look-Up-Tables}
\acro{UART}{Universal Asynchronous Receiver Transmitter}
\acro{ALU}{Arithmetic logic unit}
\acro{PC}{Program Counter}

\end{acronym}


% Abbildungsverzeichnis
\listoffigures

% Tabellenverzeichnis
\listoftables

% Selbstständigkeitserklärung
\thispagestyle{empty}
\chapter*{Selbstständigkeitserklärung}
\addcontentsline{toc}{chapter}{Selbstständigkeitserklärung}
 
Hiermit versichere ich, dass die vorliegende besondere Lernleistung
selbstständig und nur unter Verwendung der angegebenen Quellen und Hilfsmitteln
angefertigt wurde. Mir ist bekannt, dass bei nachgewiesenen Täuschungsversuchen
die Prüfung als „nicht bestanden“ erklärt werden kann. 
 
\vspace{1cm}
\titledate{Unterschrift Autor}


% Kapitel einbinden
\chapter{Einleitung}
\label{c:einleitung}
Diese Dokumentation beschreibt den Aufbau und die Funktionsweise der \ac{EPU}.
Das Projekt kam dadurch zustande, dass die Struktur und die Arbeitsweise eines
Computers, insbesondere der \ac{CPU} besser verstanden werden soll. Um dieses
Ziel zu erreichen, wurde die \ac{EPU} gebaut, da sie als lehrreicher
Mikrorechner, wobei der Hauptteil der \ac{EPU} nur aus der \ac{CPU} besteht,
die Funktionsweise und den Aufbau eines Alltagscomputer erklärt und somit
Verständnis für die Komplexität unserer heutingen Rechner einbringt.

\section{Was ist ein FPGA?}
Ein \ac{FPGA} ist ein \ac{IC}, welcher zum Aufbau digitaler Schaltungen
dient. Er besteht meist aus mehr als 100.000 Logikblöcken~\cite[S. 8]{minicpu}.
Ins Deutsche übersetzt bedeutet \ac{FPGA} soviel wie `im Feld programmierbare
[Logik-]Gatter-Anordnung', wobei `im Feld' sich dabei auf den Konsumenten
bezieht.

Die Logikschaltungen eines \ac{FPGA} sind entweder über elektronische `Schalter'
der Konfiguration entsprechend verknüpft oder es werden sogenannte \acp{LUT}
benutzt, mit denen die Logikfunktion explizit realisiert werden kann. Eine
\ac{LUT} kann verschiedene kombinatorische Funktionen (NAND, XOR, AND, NOT,
Multiplexer, etc.) aus den Eingangssignalen realisieren. Die meisten \acp{LUT}
besitzen zwischen 4 und 6 Eingangssignale. Es ist auch möglich, mehrere
\acp{LUT} in Serie zu schalten und die Limitierung durch die Eingangssignale zu
verhindern~\cite{FPGA_Aufbau}.

Da der \ac{FPGA} aber nach Verlust des Stromanschlusses die Konfiguration der
Logikelemente nicht von selbst speichert, wird meist zusätzlich noch ein
Flash-Speicher verbaut, damit nach einem Stromverlust die alte Konfiguration
wieder neu geladen werden kann. Dies hat auch den Vorteil, dass dadurch der
Status des \ac{FPGA} zurückgesetzt wird und somit ein `Neustart' schnell möglich
ist.

Anders als bei üblicher Programmierung von Computern kann bei einem \ac{FPGA}
keine herkömmliche Programmiersprache verwendet. Das `Programmieren' wird
üblicherweise als Konfiguration bezeichnet und wird mithilfe einer
Hardwarebeschreibungssprache wie z.B. VHDL oder Verilog erledigt.
\section{Beschreibung des genutzten FPGA und Entwicklungsboards}
\section{Einführung in VHDL}

\chapter{Hardware}
\label{c:hardwar}
\section{Aufbau}
\subsection{Steuerwerk}
\pagebreak
\subsection{Rechenwerk}
\pagebreak
\subsection{Dekodierer}
\pagebreak
\subsection{Programmzähler}
\pagebreak
\subsection{Stack}
\pagebreak
\subsection{Arbeitsspeicher}
\pagebreak
\subsection{Registerbelegung}
Die \ac{EPU} besitzt 16 Register, welche durch Selektion von $\log_2(16) = 4$
Adressbits angesprochen werden. Mithilfe der Tabelle~\ref{tab:registerbelegung}
soll eine Übersicht aller Register dargestellt werden.

\begin{table}[h]
\centering
\begin{tabular}{lll}
\toprule
Selektion & Name & Zweck\\
\midrule
0000 & R0  & Akkumulator\\
0001 & R1  & Allgemeine Verwendung\\
0010 & R2  & Laufvariable\\
0011 & R3  & Datenregister\\
0100 & R4  & Allgemeine Verwendung\\
0101 & R5  & Allgemeine Verwendung\\
0110 & R6  & Allgemeine Verwendung\\
0111 & R7  & Allgemeine Verwendung\\
1000 & R8  & Allgemeine Verwendung\\
1001 & R9  & Allgemeine Verwendung\\
1010 & R10 & Allgemeine Verwendung\\
1010 & R11 & Allgemeine Verwendung\\
1100 & R12 & Allgemeine Verwendung\\
1101 & R13 & Allgemeine Verwendung\\
1110 & FLA & Flagregister\\
1111 & ID  & Interruptdaten\\
\bottomrule
\end{tabular}
\caption{Registerbelegung}
\label{tab:registerbelegung}
\end{table}
\section{Befehlssatz}
\section{Ein- und Ausgabe}
\subsection{Eingabe}
\subsection{Ausgabe}

\chapter{Implementierung in VHDL}
\label{c:vhdl}
\section{Einführung in VHDL}
In diesem Abschnitt soll eine kurze Einführung in VHDL gegeben werden, damit die
nachfolgende Erklärung der Implementierung leichter zu verstehen ist.

\subsection{Was ist VHDL}
VHDL ist eine sogenannte Hardware Description Language (HDL), mit welcher es möglich ist, digitale
Schaltungen in Form von Quellcode zu schreiben, um diesen dann von einem
Computer zu Hardware weiterverarbeiten zu lassen. VHDL ist eine Abkürzung für
VHSIC HDL (\acl{VHDL}). Der geschriebene VHDL-Code wird dann synthesiert zu
einer Netzliste, welche die einzelnen Verbindungen zwischen den Komponenten
beschreibt. Diese Netzliste wird daraufhin für den jeweiligen \ac{FPGA}
übersetzt, sodass als letzter Schritt eine Konfigurationsdatei erstellt werden
kann, welche auf den \ac{FPGA} geladen werden kann.

Eine Besonderheit bei VHDL ist, dass nicht jeder geschriebene Quellcode
synthesierbar ist, was bedeutet, dass mancher Quellcode nur in der Simulation
funktioniert. In dem Umfang dieses Projektes ist aber der geschriebene Quellcode
vollkommen synthesierbar und soll nicht nur in der Simulation funktionieren.

\subsection{Notation}
Allgemein endet jede Anweisung immer mit einem Semikolon. Kommentare werden mit
einem doppelten Bindestrich (-{}-) eingeleitet und sind gültig bis zum Ende der
Zeile. Einzelne Bitwerte werden mit einfachen Anführungszeichen und Werte
von Bitvektoren (mehreren Bits) mit doppelten Anführungszeichen umrahmt.

Außerdem gibt es noch mehrere Sprachelemente, welche häufig Verwendung
haben. Eine (nicht vollständige) Beschreibung von häufig genutzten
Sprachelementen:

\begin{description}[align=right, labelwidth=1.6cm]
\item[signal]		Verbindung zweier Module. Beschreibt die Leitung zwischen den
					Modulen.
\item[variable]		Ein Zwischenspeicher für Werte, welcher \textbf{nicht} als
					Signal übersetzt wird.
\item[entity]		Schlüsselwort zur Deklaration eines Moduls (logische Einheit).
\item[in]			Deklaration eines Eingabesignals eines Moduls.
\item[out]			Deklaration eines Ausgabesignals eines Moduls.
\item[architecture]	Schlüsselwort zur Definition eines Moduls
\item[component]	Signatur eines Moduls. Wird verwendet, um ein Modul
					innerhalb eines anderen Moduls zu verwenden.
\item[package]		Eine Sammlung (Bibliothek) von \textbf{components}.
\end{description}
\pagebreak
\section{Beschreibung wichtiger Module}
\subsection{top -- Verbindung der Ein- und Ausgänge}
Das Topmodul dient als oberste Ebene, welche die \ac{EPU} mit der `Außenwelt'
verbindet. Die dazugehörigen Ein- und Ausgangssignale werden in der Datei
\textit{top.vhdl} festgelegt. Abbildung~\ref{code:top} zeigt die Definition des
Topmoduls.
\begin{figure}[htb]
\lstinputlisting[language=vhdl, tabsize=4, firstline=10, lastline=29]{../vhdl/top/top.vhdl}
\caption{Topmodul}
\label{code:top}
\end{figure}
\subsection{core -- Topmodul der CPU}
\pagebreak
\subsection{memory\_control -- Speichercontroller}
\pagebreak

\captionsetup[figure]{justification=justified,singlelinecheck=false}
\chapter{Programmstruktur}
In diesem Kapitel soll eine Übersicht über die Struktur und den Aufbau der
Programme, welche die Software der EPU bilden, dargestellt werden. Die
Voraussetzung für das Verständnis der Umsetzung der Software in die Hardware
ist bereits in Kapitel~\ref{c:hardware}, oder genauer bei der Erklärung des
Befehlssatzes in Abschnitt~\ref{s:befehlssatz}, geschehen.
\section{Einführung in EPU-Assembly}
Zur Übersetzung der eigenen Assembly-Sprache in Maschinencode, welche die CPU
verstehen, wurde ein eigener Assembler entwickelt. Der Assembler ist in  der
Interpretersprache Python geschrieben und erlaubt die Kompilierung von einer
.easm-Datei, wobei eine Vorlage mit vordefinierten Funktionen als externe Datei
auch noch übergeben werden kann. Die Ausgabe erfolgt durch das Schreiben der
einzelnen Bytes des Maschinencodes in eine Datei im .coe-Format. Der Grund für
das .coe-Format ist, dass dieses genutzt werden kann, um den RAM-Block der EPU
zu initialisieren und somit das Programm direkt ausgeführt werden kann.

Der Aufbau der Befehle, welche der Assembler versteht, wurde recht simpel
gehalten, um eine leicht zu verstehende Sprache zu erstellen, wobei viele
Ähnlichkeiten zur x86-Assembly (genutzt ist den meisten Desktopcomputern
heutzutage) bestehen, da somit das anfängliche Verstehen der Befehle leichter
wird. Der Aufbau eines Befehls richtet sich nach folgender Regel, wobei in
eckigen Klammern eingerahmte Parameter optional sind bzw. nur bei bestimmten
Befehlen notwendig sind:
\begin{lstlisting}[basicstyle=,numbers=none]
	Mnemonic[.Option] [Operand1] [,Operand2] [,Operand3]
\end{lstlisting}

Um einen Befehl aufzurufen, wird das obere Schema verwendet, was bedeutet, dass
der Befehl selbst mit seiner \textbf{Mnemonic} startet. Das Wort Mnemonic
bedeutet auf Deutsch so viel wie 'Merkspruch'. Es wird meist bei
Assemblersprachen verwendet, um beim Programmieren das Auswendiglernen des
Maschinencodes jedes Befehls zu sparen. An die Mnemonic kann eine Option nach
einem Punkt angehängt werden. Dies ist befehlsspezifisch und muss für jeden
Befehl nachgeschlagen werden. Je nach Befehl werden auch ein bis zwei weitere
Operanden benötigt, welche selbst noch in verschiedenen Formen vorkommen können
(abhängig von der gewählten Option). Um dies zu verdeutlichen, sollen nun ein
paar Beispiele das Verständnis für den Befehlsaufbau gezeigt werden.
\begin{figure}
\begin{lstlisting}[basicstyle=]
	load r1, 0xAFFE
	jmp.i $test

	data:
		.data 0xFF

	test:
		addi r2, r1, 0d10
		load r1, $data
		write.l r1, r2, 0x0
\end{lstlisting}
\caption{Befehlsaufbau - Assembly}
\label{code:instruction_code}
\end{figure}

In Abbildung ~\ref{code:instruction_code} wird zuerst in Zeile 1 eine
hexadezimale Konstante (Präfix '0x' steht für hexadezimal) in das Register R1
geladen. Danach wird ein Sprungbefehl mit der Option 'i' ausgeführt, wobei das
'i' für 'Immediate' bzw. Konstante steht und damit die Adresse, zu welcher
gesprungen werden soll, im Befehl vorhanden ist.  Die Konstant ist hier mit dem
Adressoperator '\$' vorangestellt, welcher die Adresse des \textbf{Labels}, hier
test' (Zeile 7), enthält. Dadurch wird die Ausführung in Zeile 8, also direkt
nach der Definition des Labels, fortgeführt.

Davor ist aber noch ein in Zeile 5 eine sogennante \textbf{Assemblerdirektive}
zu sehen, welche an dem '.' vor der Mnemonic zu erkennen ist.
Assemblerdirektiven sind Anweisungen an den Assembler und werden nicht direkt in
Maschinencode umgewandelt, sondern getrennt behandelt. In diesem Beispiel ist
die Assemblerdirektive 'data' verwendet worden, welche die angegebene Konstante
im ersten Operanden unmittelbar in die Byteausgabe des Maschinencodes übernimmt.
So ist es beispielsweise möglich, dass vordefinierte Variablen bereits einen
Wert beim Start des Programms zugewiesen bekommen können.

Zurück bei der Ausführung des Beispieles in Zeile 8 wird mit dem Befehl 'addi'
eine Konstante zu dem Register R1 (zweiter Operand) eine dezimale Konstante
(Präfix '0d' für dezimale Konstanten) addiert. Das Ergebnis wird im ersten
Operand gespeichert - hier R2. In Zeile 9 wird die Adresse des 'data'-Labels in
das Register R1 geladen und danach in Zeile 10 wird an die Speicheradresse in R1
der Wert im Register R2 geschrieben. Der dritte Operand wird hier nicht benötigt
und daher null gesetzt, kann aber für einen Offset der Speicheradresse dienen.
\pagebreak
\section{Funktionsaufruf}
In diesem Abschnitt soll der Aufruf von Funktonen demonstriert werden. Bei dem
Aufbau wurde eine ähnliche Methode wie beim x86-Befehlssatz verwendet. Die
einfachste Methode des Funktionsaufrufs ist das Aufrufen einer Funktion ohne
Parameter, wie unten in Abbildung~\ref{code:function_struct}  zu sehen ist.
\begin{figure}[htb]
\begin{lstlisting}[basicstyle=]
funktion:
	ret

_start:
	call.i $funktion
\end{lstlisting}
\caption{Funktionsaufruf ohne Parameter}
\label{code:function_struct}
\end{figure}
Dabei wird einfach der Befehl 'call' aufgerufen mit dem ersten Operanden hier
als Adresse zum Label 'funktion'. Nachdem die Funktion mit dem Befehl 'ret'
wieder zurückkehrt, wird automatisch der nächste Befehl nach dem Funktonsaufruf
ausgeführt.

Damit beim Zurückkehren von einer Funktion die Rücksprungadresse für den
nächsten Befehl gefunden werden kann, wird beim Aufruf von 'call' nicht nur ein
Sprungbefehl zur Funktion aufgerufen, sondern auch die Rückkehradresse auf den
Stack (siehe~\ref{s:stack}) gelegt. Beim Ausführen von 'ret' wird dann einfach
das Element vom Stack heruntergenommen und in den PC (siehe~\ref{s:pc})
geschrieben.

Bei Funktionen mit Parametern und Rückgabewert ist der Aufbau ähnlich. Das
folgende Beispiel in Abbildung~\ref{code:function_parameter} soll einen solchen
Funktionsaufruf illustrieren:
\begin{figure}[!htb]
\begin{lstlisting}[basicstyle=]
sub:
	pop r15
	pop r0
	pop r1
	push r15

	sub.u r0, r0, r1

	pop r15
	push r0
	push r15
	ret

_start:
	load r4, 0x0005
	load r5, 0x0002
	push r5
	push r4
	call.i $sub
	pop r3
\end{lstlisting}
\caption{Funktionsaufruf mit Paramtern}
\label{code:function_parameter}
\end{figure}

Zuerst werden die zu übergebenen Argumente in beliebige Register gespeichert
(Zeile 15-16) und dann in \textbf{umgekehrter} Reihenfolge auf den Stack gelegt
(Zeile 17-18). Danach findet der eigentliche Funktionsaufruf statt und die
Funktion wird aufgerufen, wobei unter anderem die Rückkehradresse wird auch auf
dem Stack abgelegt. Da die Rückkehradresse ganz oben auf dem Stack abgelegt ist,
muss diese erst vom Stack heruntergenommen und in einem Register
zwischengespeichert werden, damit die Argumente vom Stack geholt werden können
(Zeile 2). Da sie in umgekehrter Reihenfolge auf den Stack gelegt wurden, dreht
sich beim Holen der Argumente die Reihenfolge wieder um, sodass die Reihenfolge
wieder richtig ist (Zeile 3-4). Danach kann die Rückkehradresse wieder zurück
auf den Stack gelegt werden (Zeile 5). Danach kann die eigentliche Operation der
Funktion durchgeführt werden; dies ist hier die Subtraktion (Zeile 7). Da nun
der Rückgabewert auf dem Stack abgelegt werden soll, aber auch zugriffsbereit
für die aufrufende Funktion, muss zuerst wieder die Rückkehradresse vom Stack
entfernt (Zeile 9) und dann der Rückgabewert auf den Stack gelegt werden (Zeile
10). Danach kann die Rückkehradresse wieder auf den Stack (Zeile 11) und die
Funktion kann wie gewohnt zurückkehren (Zeile 12).

Durch die Nutzung des Stacks für den Rückgabewert ist es theoretisch möglich,
mehrere Rückgabewert einer Funktion zuzuweisen. In dem Beispiel ist dies aber
nicht der Fall.  Dennoch ist zu erwähnen, dass über die Register auch ein oder
mehrere Rückgabewerte theoretisch erfolgen können. Genauere Informationen sind
der jeweiligen Funktion, welche man aufrufen möchte, zu entnehmen.
\clearpage
\section{Vordefinierte Funktonen}
Einige allgemein nützige Funktionen sind bereits vorgeschrieben worden und
können als Vorlage an den Assembler übergeben werden. Dadurch können diese
vordefinierten Funktionen in allen Programmen genutzt werden. Unter anderem sind
folgende Funktionen verfügbar:
\begin{description}
\item [\_print(*data, start)] Gebe eine null-terminierte Zeichenkette an der
Stelle \textbf{data} auf dem Bildschirm an Startpositon \textbf{start}
aus.
\item [\_setcursor(pos)] Setze den Bildschirmcursor an Position \textbf{pos},
	wobei der Wert 0xAABB die x-Koordinate 0xAA und die y-Koordinate 0xBB setzt.
\item [\_getinput] Liest ein Byte vom Benutzer ein und gibt es zurück
\item [\_getstring] Liest einen 'Enter'-terminierten String vom Benutzer ein und
	gibt diesen zurück
\end{description}
Die Benutzung dieser Vorlage wird empfohlen, ist aber keine Voraussetzung. Denn
die Vorlage definiert auch, dass der Start des eigentlichen Programms beim Label
'\_start' stattfindet und dieser wird auch von der Vorlage aufgerufen. Wie
wahrscheinlich bereits aufgefallen ist, starten bei der Erstellung der Vorlage
alle genutzten Labels mit einem Unterstrich, was den Grund hat, alle
vordefinierten Funktionen und die Funktionen des Programms visuell zu trennen.
Daher ist es auch nicht empfohlen, dass im Programm Labels mit einem Unterstrich
starten.
\pagebreak
\section{Debugging}
Mithilfe des Debugging ist es möglich, Fehler in Programmen zu erkennen und
diese auch zu beseitigen. Damit dies auch bei der \ac{EPU} möglich ist, soll
dieser Abschnitt beispielhaft anhand eines Signal-Zeit-Diagrammes zeigen, wie die
Ausführung eines Befehls aussehen sollte.
Das auszuführende Befehl sei folgender, wobei angenommen wird das Register R14
bereits den Wert 0x1234 hat:
\begin{lstlisting}[basicstyle=,numbers=none]
	push r14
\end{lstlisting}

\captionsetup[figure]{justification=centering,singlelinecheck=false}
\begin{figure}[htb]
\raggedright
\begin{wave}{11}{9}{0}
 \nextwave{en\_fetch} \bit{1}{1} \bit{0}{1} \bit{1}{1} \bit{0}{1} \bit{1}{1}
					 \bit{0}{5}
 \nextwave{en\_decode} \bit{0}{1} \bit{1}{1} \bit{0}{1} \bit{1}{1} \bit{0}{1}
					   \bit{1}{1} \bit{0}{4}
 \nextwave{en\_regread} \bit{0}{6} \bit{1}{1} \bit{0}{3}
 \nextwave{en\_alu} \bit{0}{7} \bit{1}{1} \bit{0}{2}
 \nextwave{en\_stack} \bit{0}{8} \bit{1}{1} \bit{0}{1}
 \nextwave{en\_regwrite} \bit{0}{9} \bit{1}{1}
 \nextwave{pc} \known{0x0}{2} \known{0x1}{2} \known{0x2}{6}
 \nextwave{dataA} \unknown{7} \known{0x1234}{3}
 \nextwave{stack\_i\_en} \bit{0}{8} \bit{1}{1} \bit{0}{1}
 \nextwave{stack\_i\_we} \bit{0}{8} \bit{1}{1} \bit{0}{1}
\end{wave}
\caption{Signal-Zeit-Diagramm}
\label{wave:program}
\end{figure}

Wie in Abbildung~\ref{wave:program} zu sehen, beschreiben die ersten 6 Signale
die einzelnen Zustände des Steuerwerks (siehe~\ref{s:control}), danach folgt der
Programmzähler (siehe~\ref{s:pc}) und der Wert des übergebenen Registers Ra,
gefolgt von dem 'Enable'- und 'WriteEnable'-Eingang des Stacks
(siehe~\ref{s:stack}).

Im ersten Abschnitt von t=0 bis t=6 ist zu sehen, wie byteweise der Befehl
gelesen und dekodiert wird. Dabei wird bei jeder positiven Flanke des Signals
'en\_fetch' der Programmzähler inkrementiert, damit er auf das nächste zu
dekodierende Byte zeigt. Im zweiten Abschnitt von t=6 bis t=9 werden zuerst die
Werte der übergebenen Register gelesen und sind dann nach der negativen Flanke
von 'en\_regread' verfügbar (siehe Signal 'dataA'). Als nächstes wird das
Rechenwerk aktiviert und legt in diesem Fall nur fest, dass kein Sprungbefehl
ausgeführt werden muss.Danach wird nun 'en\_stack' auf 1 gesetzt (t=8) und damit
gleichzeit auch die 'Enable'- und 'Write-Enable'-Eingänge des Stacks. Der Stack
nimmt daraufhin den Wert von 'dataA' und schreibt ihn als oberstes Element auf
den Stapel. Im letzten Schritt (t=9) wird nun noch 'en\_regwrite', da jedoch
kein Ergebnis in ein Register geschrieben werden muss, passiert hier nichts.

\chapter{Fazit}
\label{c:fazit}
\section{Umfang und Aufwand}
Der Aufwand der besonderen Lernleistung war insgesamt deutlich höher als
anfänglich gedacht. Der Einstieg in das Thema erfolgt zügig, sodass die
Simulation des ersten Programm bereits einen Monat nach Beginn funktionierte.

Nachfolgend gab es kleinere Probleme, welche durch die Hilfe der Simulation
einfach zu lösen waren. Als es zur ersten Übertragung des Projektes auf den
\ac{FPGA} kam, trat ein größeres Problem auf. Die Projektatei für den \ac{FPGA}
konnte nicht übertagen werden, da der Hersteller nur ein Programm für Windows
zur Übertragung anbietet, aber das Projekt vollständig auf Linux erstellt wurde.
Nach gründlicher Recherche stellte sich heraus, dass jemand bereits dasselbe
Problem für ein ähnliches Board des selben Herstellers hatte und seinen Code für
ein Programm zur Übertragung der Datei auf den \ac{FPGA} veröffentlichte
~\cite{numatoloader}. Dies war eine große Hilfe, da jetzt nur noch die
Beschreibung des Boards im Code angepasst werden musste und anschließend auch
die Übertagung funktionierte.

Wenig später sollte auch der RAM-Block des \ac{FPGA} genutzt werden. Vorher
wurde für die Simulation der Speicher aus Flip-Flops erstellt, was bei größeren
Speichermengen aber nicht effizient ist. Dabei stellte sich das Problem, dass
die Dokumentation des DDR-RAM-Chips auf dem Board nicht sehr ausführlich ist und
keine große Hilfe bei der Implementierung bietet. Deshalb wurde kein DDR-RAM
verwendet, sondern der interne Speicher des \ac{FPGA}-Chip. Da nur 64 Kilobyte
Speicher für die 16-Bit-Adressierung der \ac{EPU} nötig sind, reicht dieser
interne Speicher aus und vereinfacht damit auch den Aufbau der \ac{EPU}, da kein
komplexer DDR-Speichercontroller implementiert werden musste.

Anfangs wurde überlegt, alle Grundrechenarten als eigene Befehle zu
implementieren. Nach längerer Abwegung wurde dagegen entschieden und nur die
Addition und Subtraktion in die Hardware implementiert. Der Gründ für diese
Entscheidung war, dass die anderen Rechenarten über die Software später
implementiert werden können und nicht essentiell sind. Grundsätzlich lohnte sich
nicht der Zeitaufwand einer Implementierung der Multiplikation und
Division, da diese Befehle nicht häufig genug Anwendung finden.

\section{Ziele und Lernerfolg}
Hauptziel der besonderen Lernleistung war es, einen funktionsfähigen
Mikrorechner zu produzieren und dieser Ziel ist sehr gut gelungen.

Außerdem sollte eine Ausgabe über einen Bildschirm als auch eine Eingabe über
eine Tastatur möglich sein. Beide Vorgaben sind erfüllt und funktionsfähig.

Andere anfangs übernommene Ziele, wie eine Netzwerkverbindung, Soundtreiber oder
externer Speicher über den MicroSD-Kartenslot mussten aus Zeitgründen leider
nicht umgesetzt werden. Dennoch sind diese Ziele als mögliche Erweiterungen zu
sehen, welche im Nachhinein noch implementiert werden könnten.

Der Lernerfolg bei diesem Projekt war außerordentlich hoch. Nicht nur die
Sprache VHDL und deren Syntax wurde erlernt und angewendet, sondern außerdem
wurden neue Kenntnisse über den Aufbau eines Rechners, insbesondere eines
Prozessors, gewonnen. Das Projekt erlaubte gute Einblicke sowohl in den
Hardware-Aufbau als auch in die umfangreiche Entwicklung von Hardware. Diese
Arbeit auf einer sehr niedrigen Abstraktionsebene ("'low-level"') verbesserte das
allgemeine Verständnis darüber, wie ein Rechner funktioniert. Dieses erworbene
Wissen hilft auch bei anderen Projekten, die auf einer höheren Abstraktionsebene
entwickelt werden. Besonders beim Beheben von Softwarefehlern ist dieses Wissen
einsetzbar und stellt dabei eine große Hilfe dar.

Das neuerworbene Wissen für das, was im Inneren eines Computers auf der tiefsten
Abstraktionsebene passiert, ist stark gestiegen. Ein solcher Lernerfolg wäre
ohne dieses Projekt nicht zustande gekommen. Daher ist ein großer Teil des
Lernerfolges diesem Themengebiet zuzuordnen.

\pagebreak


% Literaturverzeichnis
\bibliography{literatur}

\end{document}
