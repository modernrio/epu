% Dokumentenklasse
\documentclass[a4paper,12pt,liststotoc, parskip=half]{scrreprt}

% ============ Pakete ============
% Dokumentinformationen
\usepackage[
	pdftitle={Dokumentation EPU},	
	pdfsubject={},
	pdfauthor={Markus Schneider},
	pdfkeywords={},
	% Links nicht einrahmen
	% hidelinks
]{hyperref}

% Standardpakete
\usepackage[utf8]{inputenc}				% UTF-8 Zeichensatz
\usepackage[ngerman]{babel}				% Alle Bezeichnungen auf die deutsche Sprache anpassen
\usepackage[T1]{fontenc}				% Unterstützung für westeuropäische Codierung(Umlaute)
\usepackage{graphicx}					% Grafiken einbinden
\usepackage{subfig}						% Abbildungen und Tabellen
\graphicspath{{img/}}					% Pfad zu Grafiken
\usepackage{fancyhdr}					% Einfache Bearbeitung von Kopf- und Fußzeile
\usepackage{lmodern}					% Verändert die Schriftart auf "Latin Modern"
\usepackage{color}						% Farbenmanagement
\usepackage[printonlyused]{acronym}		% Abkürzungsverzeichnis
\usepackage{booktabs}					% Tabellen ("Publication Quality")
\usepackage{setspace}					% Abstände

% Zusätzliche Schriftzeichen der American Mathematical Society
\usepackage{amsfonts}
\usepackage{amsmath}

% ============ Kopf- und Fußzeile ============
% Style
\pagestyle{fancy}

% Kopfzeile
\lhead{}
\chead{}
\rhead{\slshape \leftmark}

% Fußzeile
\lfoot{}
\cfoot{\thepage}
\rfoot{}

% Trennlinien
\renewcommand{\headrulewidth}{0.4pt}	% Dünne Trennline nach der Kopfzeile
\renewcommand{\footrulewidth}{0pt}		% Keine Trennline vor der Fußzeile

% ============ Paketeinstellungen & Sonstiges ============
% Besondere Trennungen
\hyphenation{De-zi-mal-tren-nung}

% ============ Dokumentbeginn ============
\begin{document}

% Seiten ohne Kopf- und Fußzeilen sowie Seitenzahl
\pagestyle{empty}

% Festlegung der Art der Zitierung
\bibliographystyle{unsrtdin}

\title{Implementierung eines Mikrorechners in VHDL auf einem FPGA}
\date{\today}
\author{Markus Schneider}

\maketitle


% Beendet eine Seite und erzwingt auf den nachfolgenden Seiten die Ausgabe aller Gleitobjekte
% (z.B. Abbildungen), die bislang definiert, aber noch nicht ausgegeben wurden. Dieser Befehl fügt,
% falls nötig, eine leere Seite ein, sodaß die nächste Seite nach den Gleitobjekten eine ungerade
% Seitennummer hat. 
\cleardoubleoddpage{}

% Seiten ab jetzt mit Kopf- und Fußzeilen sowie Seitenzahl
\pagestyle{fancy}

% Inhaltsverzeichnis
\tableofcontents

% Abkürzungsverzeichnis
\begin{acronym}

\thispagestyle{empty}
\chapter*{Abkürzungsverzeichnis}
\addcontentsline{toc}{chapter}{Abkürzungsverzeichnis}
\acro{EPU}{Educational Processing Unit}
\acro{CPU}{Central Processing Unit}
\acro{FPGA}{Field-programmable gate array}

\end{acronym}


% Abbildungsverzeichnis
\listoffigures

% Tabellenverzeichnis
\listoftables

% Kapitel einbinden
\chapter{Einleitung}
\label{c:einleitung}
Diese Dokumentation beschreibt den Aufbau und die Funktionsweise der \ac{EPU}. Das Projekt kam
dadurch zustande, dass die Struktur und die Arbeitsweise eines Computers, insbesondere der
\ac{CPU} besser verstanden werden soll. Um dieses Ziel zu erreichen, wurde die \ac{EPU} gebaut, da
sie als lehrreicher Computer, wobei der Hauptteil der \ac{EPU} nur aus einer \ac{CPU} besteht, die
Funktionsweise und den Aufbau eines Alltagscomputer erklärt und somit Verständnis für die
Komplexität unserer heutingen Rechner einbringt.
\section{Was ist ein FPGA?}
\section{Beschreibung des genutzten FPGA und Entwicklungsboard}

\chapter{Hardware}
\label{c:hardwar}
\section{Aufbau}
\subsection{Steuerwerk}
\pagebreak
\subsection{Rechenwerk}
\pagebreak
\subsection{Dekodierer}
\pagebreak
\subsection{Programmzähler}
\pagebreak
\subsection{Stack}
\pagebreak
\subsection{Arbeitsspeicher}
\pagebreak
\subsection{Registerbelegung}
Die \ac{EPU} besitzt 16 Register, welche durch Selektion von $\log_2(16) = 4$
Adressbits angesprochen werden. Mithilfe der Tabelle~\ref{tab:registerbelegung}
soll eine Übersicht aller Register dargestellt werden.

\begin{table}[h]
\centering
\begin{tabular}{lll}
\toprule
Selektion & Name & Zweck\\
\midrule
0000 & R0  & Akkumulator\\
0001 & R1  & Allgemeine Verwendung\\
0010 & R2  & Laufvariable\\
0011 & R3  & Datenregister\\
0100 & R4  & Allgemeine Verwendung\\
0101 & R5  & Allgemeine Verwendung\\
0110 & R6  & Allgemeine Verwendung\\
0111 & R7  & Allgemeine Verwendung\\
1000 & R8  & Allgemeine Verwendung\\
1001 & R9  & Allgemeine Verwendung\\
1010 & R10 & Allgemeine Verwendung\\
1010 & R11 & Allgemeine Verwendung\\
1100 & R12 & Allgemeine Verwendung\\
1101 & R13 & Allgemeine Verwendung\\
1110 & FLA & Flagregister\\
1111 & ID  & Interruptdaten Verwendung\\
\bottomrule
\end{tabular}
\caption{Registerbelegung}
\label{tab:registerbelegung}
\end{table}
\clearpage
\section{Befehlssatz}
\section{Ein- und Ausgabe}
\subsection{Eingabe}
\subsection{Ausgabe}

\chapter{Implementierung in VHDL}
\label{c:implementierunginvhdl}
\section{Schemata}
\pagebreak
\section{Beschreibung wichtiger Module}
\subsection{top -- Verbindung mit der Hardware}
\pagebreak
\subsection{core -- Topmodul der CPU}
\pagebreak
\subsection{memory\_control -- Speichercontroller}
\pagebreak

\chapter{Fazit}
\label{c:fazit}
\section{Umfang und Aufwand}
\pagebreak
\section{Ziele}
\pagebreak


% Literaturverzeichnis
\bibliography{literatur}

\end{document}
