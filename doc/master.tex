% Dokumentenklasse
\documentclass[a4paper,12pt,liststotoc, parskip=half]{scrreprt}

% ============ Pakete ============
% Dokumentinformationen
\usepackage[
	pdftitle={Dokumentation EPU},	
	pdfsubject={},
	pdfauthor={Markus Schneider},
	pdfkeywords={},
	% Links nicht einrahmen
	% hidelinks
]{hyperref}

% Standardpakete
\usepackage[utf8]{inputenc}				% UTF-8 Zeichensatz
\usepackage[ngerman]{babel}				% Alle Bezeichnungen auf die deutsche Sprache anpassen
\usepackage[T1]{fontenc}				% Unterstützung für westeuropäische Codierung(Umlaute)
\usepackage{graphicx}					% Grafiken einbinden
\usepackage{subfig}						% Abbildungen und Tabellen
\graphicspath{{img/}}					% Pfad zu Grafiken
\usepackage{fancyhdr}					% Einfache Bearbeitung von Kopf- und Fußzeile
\usepackage{lmodern}					% Verändert die Schriftart auf "Latin Modern"
\usepackage{color}						% Farbenmanagement
\usepackage[printonlyused]{acronym}		% Abkürzungsverzeichnis
\usepackage{booktabs}					% Tabellen ("Publication Quality")
\usepackage{setspace}					% Abstände

% Zusätzliche Schriftzeichen der American Mathematical Society
\usepackage{amsfonts}
\usepackage{amsmath}

% ============ Kopf- und Fußzeile ============
% Style
\pagestyle{fancy}

% Kopfzeile
\lhead{}
\chead{}
\rhead{\slshape \leftmark}

% Fußzeile
\lfoot{}
\cfoot{\thepage}
\rfoot{}

% Trennlinien
\renewcommand{\headrulewidth}{0.4pt}	% Dünne Trennline nach der Kopfzeile
\renewcommand{\footrulewidth}{0pt}		% Keine Trennline vor der Fußzeile

% ============ Paketeinstellungen & Sonstiges ============
% Besondere Trennungen
\hyphenation{De-zi-mal-tren-nung}

% ============ Dokumentbeginn ============
\begin{document}

% Seiten ohne Kopf- und Fußzeilen sowie Seitenzahl
\pagestyle{empty}

% Festlegung der Art der Zitierung
\bibliographystyle{unsrt}

\title{Implementierung eines Mikrorechners in VHDL auf einem FPGA}
\date{\today}
\author{Markus Schneider}
% ----------------------------------------------------------------------------------------------------------
% Titelseite
% ----------------------------------------------------------------------------------------------------------
\thispagestyle{empty}
\begin{center}
	\includegraphics[scale=0.33]{img/bsgg.png}\\
	\vspace*{2cm}
	\Large
	\textbf{Besondere Lernleistung}\\
	\textbf{Berufliche Schulen Groß-Gerau}\\
	\vspace*{2cm}
	\Huge
	\textbf{Implementierung eines Mikrorechners in VHDL auf einem FPGA}\\
	\vspace*{0.5cm}
	\Large
	% über das Thema\\
	\vspace*{1cm}
	\textbf{Technikwissenschaft \& Technologie}\\
	\vspace*{2cm}
	
	\vfill
	\large
	\newcolumntype{x}[1]{>{\raggedleft\arraybackslash\hspace{0pt}}p{#1}}
	\begin{tabular}{x{6cm}p{7.5cm}}
		\rule{0mm}{4ex}\textbf{Autor:} & Markus Schneider \\
		\rule{0mm}{4ex}\textbf{Klasse:} & 13BGDE \\
		\rule{0mm}{4ex}\textbf{Wohnort:} & Mörfelden-Walldorf \\
		\rule{0mm}{4ex}\textbf{Prüfer:} & Uwe Homm, Friedrich Ernst \\ 
		\rule{0mm}{4ex}\textbf{Abgabedatum:} & 30.03.2017 \\ 
	\end{tabular} 
\end{center}
\pagebreak


% Beendet eine Seite und erzwingt auf den nachfolgenden Seiten die Ausgabe aller Gleitobjekte
% (z.B. Abbildungen), die bislang definiert, aber noch nicht ausgegeben wurden. Dieser Befehl fügt,
% falls nötig, eine leere Seite ein, sodaß die nächste Seite nach den Gleitobjekten eine ungerade
% Seitennummer hat. 
\cleardoubleoddpage{}

% Seiten ab jetzt mit Kopf- und Fußzeilen sowie Seitenzahl
\pagestyle{fancy}

% Inhaltsverzeichnis
\tableofcontents

% Abkürzungsverzeichnis
\begin{acronym}

\thispagestyle{empty}
\chapter*{Abkürzungsverzeichnis}
\addcontentsline{toc}{chapter}{Abkürzungsverzeichnis}
\acro{EPU}{Educational Processing Unit}
\acro{CPU}{Central Processing Unit}
\acro{FPGA}{Field-programmable gate array}
\acro{IC}{Integrated Circuit}
\acro{LUT}{Look-Up-Table}
\acrodefplural{LUT}[LUTs]{Look-Up-Tables}
\acro{UART}{Universal Asynchronous Receiver Transmitter}
\acro{ALU}{Arithmetic logic unit}
\acro{PC}{Program Counter}

\end{acronym}


% Abbildungsverzeichnis
\listoffigures

% Tabellenverzeichnis
\listoftables

% Kapitel einbinden
\chapter{Einleitung}
\label{c:einleitung}
Diese Dokumentation beschreibt den Aufbau und die Funktionsweise der \ac{EPU}.
Das Projekt kam dadurch zustande, dass die Struktur und die Arbeitsweise eines
Computers, insbesondere der \ac{CPU} besser verstanden werden soll. Um dieses
Ziel zu erreichen, wurde die \ac{EPU} gebaut, da sie als lehrreicher
Mikrorechner, wobei der Hauptteil der \ac{EPU} nur aus der \ac{CPU} besteht,
die Funktionsweise und den Aufbau eines Alltagscomputer erklärt und somit
Verständnis für die Komplexität unserer heutingen Rechner einbringt.

\section{Was ist ein FPGA?}
Ein \ac{FPGA} ist ein \ac{IC}, welcher zum Aufbau digitaler Schaltungen
dient. Er besteht meist aus mehr als 100.000 Logikblöcken~\cite[S. 8]{minicpu}.
Ins Deutsche übersetzt bedeutet \ac{FPGA} soviel wie `im Feld programmierbare
[Logik-]Gatter-Anordnung', wobei `im Feld' sich dabei auf den Konsumenten
bezieht.

Die Logikschaltungen eines \ac{FPGA} sind entweder über elektronische `Schalter'
der Konfiguration entsprechend verknüpft oder es werden sogenannte \acp{LUT}
benutzt, mit denen die Logikfunktion explizit realisiert werden kann. Eine
\ac{LUT} kann verschiedene kombinatorische Funktionen (NAND, XOR, AND, NOT,
Multiplexer, etc.) aus den Eingangssignalen realisieren. Die meisten \acp{LUT}
besitzen zwischen 4 und 6 Eingangssignale. Es ist auch möglich, mehrere
\acp{LUT} in Serie zu schalten und die Limitierung durch die Eingangssignale zu
verhindern~\cite{FPGA_Aufbau}.

Da der \ac{FPGA} aber nach Verlust des Stromanschlusses die Konfiguration der
Logikelemente nicht von selbst speichert, wird meist zusätzlich noch ein
Flash-Speicher verbaut, damit nach einem Stromverlust die alte Konfiguration
wieder neu geladen werden kann. Dies hat auch den Vorteil, dass dadurch der
Status des \ac{FPGA} zurückgesetzt wird und somit ein `Neustart' schnell möglich
ist.

Anders als bei üblicher Programmierung von Computern kann bei einem \ac{FPGA}
keine herkömmliche Programmiersprache verwendet. Das `Programmieren' wird
üblicherweise als Konfiguration bezeichnet und wird mithilfe einer
Hardwarebeschreibungssprache wie z.B. VHDL oder Verilog erledigt.
\section{Beschreibung des genutzten FPGA und Entwicklungsboards}
\section{Einführung in VHDL}

\chapter{Hardware}
\label{c:hardwar}
\section{Aufbau}
\subsection{Steuerwerk}
\pagebreak
\subsection{Rechenwerk}
\pagebreak
\subsection{Dekodierer}
\pagebreak
\subsection{Programmzähler}
\pagebreak
\subsection{Stack}
\pagebreak
\subsection{Arbeitsspeicher}
\pagebreak
\subsection{Registerbelegung}
Die \ac{EPU} besitzt 16 Register, welche durch Selektion von $\log_2(16) = 4$
Adressbits angesprochen werden. Mithilfe der Tabelle~\ref{tab:registerbelegung}
soll eine Übersicht aller Register dargestellt werden.

\begin{table}[h]
\centering
\begin{tabular}{lll}
\toprule
Selektion & Name & Zweck\\
\midrule
0000 & R0  & Akkumulator\\
0001 & R1  & Allgemeine Verwendung\\
0010 & R2  & Laufvariable\\
0011 & R3  & Datenregister\\
0100 & R4  & Allgemeine Verwendung\\
0101 & R5  & Allgemeine Verwendung\\
0110 & R6  & Allgemeine Verwendung\\
0111 & R7  & Allgemeine Verwendung\\
1000 & R8  & Allgemeine Verwendung\\
1001 & R9  & Allgemeine Verwendung\\
1010 & R10 & Allgemeine Verwendung\\
1010 & R11 & Allgemeine Verwendung\\
1100 & R12 & Allgemeine Verwendung\\
1101 & R13 & Allgemeine Verwendung\\
1110 & FLA & Flagregister\\
1111 & ID  & Interruptdaten\\
\bottomrule
\end{tabular}
\caption{Registerbelegung}
\label{tab:registerbelegung}
\end{table}
\section{Befehlssatz}
\section{Ein- und Ausgabe}
\subsection{Eingabe}
\subsection{Ausgabe}

\chapter{Implementierung in VHDL}
\label{c:vhdl}
\section{Einführung in VHDL}
In diesem Abschnitt soll eine kurze Einführung in VHDL gegeben werden, damit die
nachfolgende Erklärung der Implementierung leichter zu verstehen ist.

\subsection{Was ist VHDL}
VHDL ist eine sogenannte Hardware Description Language (HDL), mit welcher es möglich ist, digitale
Schaltungen in Form von Quellcode zu schreiben, um diesen dann von einem
Computer zu Hardware weiterverarbeiten zu lassen. VHDL ist eine Abkürzung für
VHSIC HDL (\acl{VHDL}). Der geschriebene VHDL-Code wird dann synthesiert zu
einer Netzliste, welche die einzelnen Verbindungen zwischen den Komponenten
beschreibt. Diese Netzliste wird daraufhin für den jeweiligen \ac{FPGA}
übersetzt, sodass als letzter Schritt eine Konfigurationsdatei erstellt werden
kann, welche auf den \ac{FPGA} geladen werden kann.

Eine Besonderheit bei VHDL ist, dass nicht jeder geschriebene Quellcode
synthesierbar ist, was bedeutet, dass mancher Quellcode nur in der Simulation
funktioniert. In dem Umfang dieses Projektes ist aber der geschriebene Quellcode
vollkommen synthesierbar und soll nicht nur in der Simulation funktionieren.

\subsection{Notation}
Allgemein endet jede Anweisung immer mit einem Semikolon. Kommentare werden mit
einem doppelten Bindestrich (-{}-) eingeleitet und sind gültig bis zum Ende der
Zeile. Einzelne Bitwerte werden mit einfachen Anführungszeichen und Werte
von Bitvektoren (mehreren Bits) mit doppelten Anführungszeichen umrahmt.

Außerdem gibt es noch mehrere Sprachelemente, welche häufig Verwendung
haben. Eine (nicht vollständige) Beschreibung von häufig genutzten
Sprachelementen:

\begin{description}[align=right, labelwidth=1.6cm]
\item[signal]		Verbindung zweier Module. Beschreibt die Leitung zwischen den
					Modulen.
\item[variable]		Ein Zwischenspeicher für Werte, welcher \textbf{nicht} als
					Signal übersetzt wird.
\item[entity]		Schlüsselwort zur Deklaration eines Moduls (logische Einheit).
\item[in]			Deklaration eines Eingabesignals eines Moduls.
\item[out]			Deklaration eines Ausgabesignals eines Moduls.
\item[architecture]	Schlüsselwort zur Definition eines Moduls
\item[component]	Signatur eines Moduls. Wird verwendet, um ein Modul
					innerhalb eines anderen Moduls zu verwenden.
\item[package]		Eine Sammlung (Bibliothek) von \textbf{components}.
\end{description}
\pagebreak
\section{Beschreibung wichtiger Module}
\subsection{top -- Verbindung der Ein- und Ausgänge}
Das Topmodul dient als oberste Ebene, welche die \ac{EPU} mit der `Außenwelt'
verbindet. Die dazugehörigen Ein- und Ausgangssignale werden in der Datei
\textit{top.vhdl} festgelegt. Abbildung~\ref{code:top} zeigt die Definition des
Topmoduls.
\begin{figure}[htb]
\lstinputlisting[language=vhdl, tabsize=4, firstline=10, lastline=29]{../vhdl/top/top.vhdl}
\caption{Topmodul}
\label{code:top}
\end{figure}
\subsection{core -- Topmodul der CPU}
\pagebreak
\subsection{memory\_control -- Speichercontroller}
\pagebreak

\chapter{Fazit}
\label{c:fazit}
\section{Umfang und Aufwand}
Der Aufwand der besonderen Lernleistung war insgesamt deutlich höher als
anfänglich gedacht. Der Einstieg in das Thema erfolgt zügig, sodass die
Simulation des ersten Programm bereits einen Monat nach Beginn funktionierte.

Nachfolgend gab es kleinere Probleme, welche durch die Hilfe der Simulation
einfach zu lösen waren. Als es zur ersten Übertragung des Projektes auf den
\ac{FPGA} kam, trat ein größeres Problem auf. Die Projektatei für den \ac{FPGA}
konnte nicht übertagen werden, da der Hersteller nur ein Programm für Windows
zur Übertragung anbietet, aber das Projekt vollständig auf Linux erstellt wurde.
Nach gründlicher Recherche stellte sich heraus, dass jemand bereits dasselbe
Problem für ein ähnliches Board des selben Herstellers hatte und seinen Code für
ein Programm zur Übertragung der Datei auf den \ac{FPGA} veröffentlichte
~\cite{numatoloader}. Dies war eine große Hilfe, da jetzt nur noch die
Beschreibung des Boards im Code angepasst werden musste und anschließend auch
die Übertagung funktionierte.

Wenig später sollte auch der RAM-Block des \ac{FPGA} genutzt werden. Vorher
wurde für die Simulation der Speicher aus Flip-Flops erstellt, was bei größeren
Speichermengen aber nicht effizient ist. Dabei stellte sich das Problem, dass
die Dokumentation des DDR-RAM-Chips auf dem Board nicht sehr ausführlich ist und
keine große Hilfe bei der Implementierung bietet. Deshalb wurde kein DDR-RAM
verwendet, sondern der interne Speicher des \ac{FPGA}-Chip. Da nur 64 Kilobyte
Speicher für die 16-Bit-Adressierung der \ac{EPU} nötig sind, reicht dieser
interne Speicher aus und vereinfacht damit auch den Aufbau der \ac{EPU}, da kein
komplexer DDR-Speichercontroller implementiert werden musste.

Anfangs wurde überlegt, alle Grundrechenarten als eigene Befehle zu
implementieren. Nach längerer Abwegung wurde dagegen entschieden und nur die
Addition und Subtraktion in die Hardware implementiert. Der Gründ für diese
Entscheidung war, dass die anderen Rechenarten über die Software später
implementiert werden können und nicht essentiell sind. Grundsätzlich lohnte sich
nicht der Zeitaufwand einer Implementierung der Multiplikation und
Division, da diese Befehle nicht häufig genug Anwendung finden.

\section{Ziele und Lernerfolg}
Hauptziel der besonderen Lernleistung war es, einen funktionsfähigen
Mikrorechner zu produzieren und dieser Ziel ist sehr gut gelungen.

Außerdem sollte eine Ausgabe über einen Bildschirm als auch eine Eingabe über
eine Tastatur möglich sein. Beide Vorgaben sind erfüllt und funktionsfähig.

Andere anfangs übernommene Ziele, wie eine Netzwerkverbindung, Soundtreiber oder
externer Speicher über den MicroSD-Kartenslot mussten aus Zeitgründen leider
nicht umgesetzt werden. Dennoch sind diese Ziele als mögliche Erweiterungen zu
sehen, welche im Nachhinein noch implementiert werden könnten.

Der Lernerfolg bei diesem Projekt war außerordentlich hoch. Nicht nur die
Sprache VHDL und deren Syntax wurde erlernt und angewendet, sondern außerdem
wurden neue Kenntnisse über den Aufbau eines Rechners, insbesondere eines
Prozessors, gewonnen. Das Projekt erlaubte gute Einblicke sowohl in den
Hardware-Aufbau als auch in die umfangreiche Entwicklung von Hardware. Diese
Arbeit auf einer sehr niedrigen Abstraktionsebene ("'low-level"') verbesserte das
allgemeine Verständnis darüber, wie ein Rechner funktioniert. Dieses erworbene
Wissen hilft auch bei anderen Projekten, die auf einer höheren Abstraktionsebene
entwickelt werden. Besonders beim Beheben von Softwarefehlern ist dieses Wissen
einsetzbar und stellt dabei eine große Hilfe dar.

Das neuerworbene Wissen für das, was im Inneren eines Computers auf der tiefsten
Abstraktionsebene passiert, ist stark gestiegen. Ein solcher Lernerfolg wäre
ohne dieses Projekt nicht zustande gekommen. Daher ist ein großer Teil des
Lernerfolges diesem Themengebiet zuzuordnen.

\pagebreak


% Literaturverzeichnis
\bibliography{literatur}

\end{document}
