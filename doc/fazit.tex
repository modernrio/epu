\chapter{Fazit}
\label{c:fazit}
\section{Umfang und Aufwand}
Der Aufwand der besonderen Lernleistung war insgesamt deutlich höher als
anfänglich gedacht. Der Einstieg in das Thema ging ziemlich schnell, sodass die
Simulation des ersten Programm bereits ein Monat nach Beginn funktionierte.

Nachfolgend gab es kleinere Probleme, bei denen die Behebung durch die
Simulation deutlich vereinfacht wurde. Als es dann zur ersten Übertragung des
Projektes auf den \ac{FPGA} kam, trat ein größeres Problem auf. Zuerst konnte die
Datei für den \ac{FPGA} nicht übertagen werden, da der Hersteller nur ein Programm
für Windows zur Übertragung anbietete, aber der Rest des Projektes bereits
vollständig auf Linux erstellt wurde. Nach etwas Recherche stellte sich dann
aber raus, dass jemand bereits das selbe Problem für ein ähnliches Board des
selben Herstellers hatte und seinen Code für ein Programm zur Übertragung
veröffentlicht hatte~\cite{numatoloader}. Dies war eine große Hilfe, da nun nur noch die
Beschreibung des Boards im Code angepasst werden musste und somit auch die
Übertagung funktionierte.

Wenig später sollte auch der RAM-Block des \ac{FPGA} genutzt werden, wobei
vorher nur für die Simulation der Speicher aus Flip-Flops erstellt wurde. Dabei
stellte sic das Problem, dass die Dokumentation des DDR-RAM-Chips auf dem Board
nicht sehr ausführlich ist und keine große Hilfe bei der Implementierung bietet.
Deshalb wurde kein DDR-RAM verwendet, sondern der interne Speicher des
\ac{FPGA}-Chip. Da nur 64 Kilobyte an Speicher für die 16-Bit-Adressierung der
\ac{EPU} nötig sind, reicht dieser interne Speicher vollkommen aus und
vereinfacht damit auch den Aufbau der \ac{EPU}, da kein komplexer
DDR-Speichercontroller implementiert werden musste.

Auch wurde anfangs überlegt, alle Grundrechenarten als eigene Befehle zu
implementieren. Nach langer Überlegung wurde aber dagegen entschieden und nur
die Addition und Subtraktion in die Hardware implementiert, da die anderen
Rechenarten über die Software später implementiert werden können und nicht
essentiell sind. Grundsätzlich lohnte sich nicht der notwendige Zeitaufwand der
Implementierung für die Multiplikation und Division, da diese Befehle nicht
häufig genug Verwendung bekommen.

\section{Ziele und Lernerfolg}
Hauptziel der besonderen Lernleistung war es, einen funktionsfähigen
Mikrorechner bzw. zu produzieren und dieser Ziel ist auch sehr gut gelungen.
Außerdem sollte eine Ausgabe über einen Bildschirm als auch eine Eingabe über
eine Tastatur möglich sein, was beides funktioniert.

Andere anfangs sehr übernommene Ziele, wie eine Netzwerkverbindung, Soundtreiber
oder externer Speicher über den MicroSD-Kartenslot, mussten aus Zeitgründen
leider gestrichen werden. Dennoch sind diese Ziele als mögliche Erweiterungen zu
sehen, welche im Nachhinein noch implementiert werden könnten.

Der Lernerfolg bei diesem Projekt war außerordentlich hoch. Nicht nur die
Sprache VHDL und deren Syntax wurde gelernt und angewendet, sondern viel mehr
wurde über den Aufbau eines Rechnes, insbesondere eines Prozessors, gelernt.
Auch wie Hardware entwickelt wird und worauf dabei zu achten ist, darf nicht
unterschätzt werden. Diese Arbeit auf einer sehr niedrigen Ebene ('low-level')
hatte sehr viele Vorteile und verbesserte das allgemeine Verständnis, wie ein
Rechner funktioniert und hilft auch bei anderen Projekten, die nicht so nah an
der Hardware liegen. Besonders beim Beheben von Fehlern in Software ist dieses
Wissen nicht ersetzbar und stellt dabei eine große Hilfe dar.
\pagebreak
