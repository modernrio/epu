\chapter{Fazit}
\label{c:fazit}
\section{Umfang und Aufwand}
Der Aufwand der besonderen Lernleistung war insgesamt deutlich höher als
anfänglich gedacht. Der Einstieg in das Thema erfolgt zügig, sodass die
Simulation des ersten Programm bereits einen Monat nach Beginn funktionierte.

Nachfolgend gab es kleinere Probleme, welche durch die Hilfe der Simulation
einfach zu lösen waren. Als es zur ersten Übertragung des Projektes auf den
\ac{FPGA} kam, trat ein größeres Problem auf. Die Projektatei für den \ac{FPGA}
konnte nicht übertagen werden, da der Hersteller nur ein Programm für Windows
zur Übertragung anbietet, aber das Projekt vollständig auf Linux erstellt wurde.
Nach gründlicher Recherche stellte sich heraus, dass jemand bereits dasselbe
Problem für ein ähnliches Board des selben Herstellers hatte und seinen Code für
ein Programm zur Übertragung der Datei auf den \ac{FPGA} veröffentlichte
~\cite{numatoloader}. Dies war eine große Hilfe, da jetzt nur noch die
Beschreibung des Boards im Code angepasst werden musste und anschließend auch
die Übertagung funktionierte.

Wenig später sollte auch der RAM-Block des \ac{FPGA} genutzt werden. Vorher
wurde für die Simulation der Speicher aus Flip-Flops erstellt, was bei größeren
Speichermengen aber nicht effizient ist. Dabei stellte sich das Problem, dass
die Dokumentation des DDR-RAM-Chips auf dem Board nicht sehr ausführlich ist und
keine große Hilfe bei der Implementierung bietet. Deshalb wurde kein DDR-RAM
verwendet, sondern der interne Speicher des \ac{FPGA}-Chip. Da nur 64 Kilobyte
Speicher für die 16-Bit-Adressierung der \ac{EPU} nötig sind, reicht dieser
interne Speicher aus und vereinfacht damit auch den Aufbau der \ac{EPU}, da kein
komplexer DDR-Speichercontroller implementiert werden musste.

Anfangs wurde überlegt, alle Grundrechenarten als eigene Befehle zu
implementieren. Nach längerer Abwegung wurde dagegen entschieden und nur die
Addition und Subtraktion in die Hardware implementiert. Der Gründ für diese
Entscheidung war, dass die anderen Rechenarten über die Software später
implementiert werden können und nicht essentiell sind. Grundsätzlich lohnte sich
nicht der Zeitaufwand einer Implementierung der Multiplikation und
Division, da diese Befehle nicht häufig genug Anwendung finden.

\section{Ziele und Lernerfolg}
Hauptziel der besonderen Lernleistung war es, einen funktionsfähigen
Mikrorechner zu produzieren und dieser Ziel ist sehr gut gelungen.

Außerdem sollte eine Ausgabe über einen Bildschirm als auch eine Eingabe über
eine Tastatur möglich sein. Beide Vorgaben sind erfüllt und funktionsfähig.

Andere anfangs übernommene Ziele, wie eine Netzwerkverbindung, Soundtreiber oder
externer Speicher über den MicroSD-Kartenslot mussten aus Zeitgründen leider
nicht umgesetzt werden. Dennoch sind diese Ziele als mögliche Erweiterungen zu
sehen, welche im Nachhinein noch implementiert werden könnten.

Der Lernerfolg bei diesem Projekt war außerordentlich hoch. Nicht nur die
Sprache VHDL und deren Syntax wurde erlernt und angewendet, sondern außerdem
wurden neue Kenntnisse über den Aufbau eines Rechners, insbesondere eines
Prozessors, gewonnen. Das Projekt erlaubte gute Einblicke sowohl in den
Hardware-Aufbau als auch in die umfangreiche Entwicklung von Hardware. Diese
Arbeit auf einer sehr niedrigen Abstraktionsebene ("'low-level"') verbesserte das
allgemeine Verständnis darüber, wie ein Rechner funktioniert. Dieses erworbene
Wissen hilft auch bei anderen Projekten, die auf einer höheren Abstraktionsebene
entwickelt werden. Besonders beim Beheben von Softwarefehlern ist dieses Wissen
einsetzbar und stellt dabei eine große Hilfe dar.

Das neuerworbene Wissen für das, was im Inneren eines Computers auf der tiefsten
Abstraktionsebene passiert, ist stark gestiegen. Ein solcher Lernerfolg wäre
ohne dieses Projekt nicht zustande gekommen. Daher ist ein großer Teil des
Lernerfolges diesem Themengebiet zuzuordnen.

\pagebreak
