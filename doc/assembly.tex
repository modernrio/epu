\captionsetup[figure]{justification=justified,singlelinecheck=false}
\chapter{Programmstruktur}
In diesem Kapitel soll eine Übersicht über die Struktur und den Aufbau der
Programme, welche die Software der EPU bilden, dargestellt werden. Die
Voraussetzung für das Verständnis der Umsetzung der Software in die Hardware
ist bereits in Kapitel~\ref{c:hardware}, oder genauer bei der Erklärung des
Befehlssatzes in Abschnitt~\ref{s:befehlssatz}, geschehen.
\section{Einführung in EPU-Assembly}
Zur Übersetzung der eigenen Assembly-Sprache in Maschinencode, welche die CPU
verstehen, wurde ein eigener Assembler entwickelt. Der Assembler ist in  der
Interpretersprache Python geschrieben und erlaubt die Kompilierung von einer
.easm-Datei, wobei eine Vorlage mit vordefinierten Funktionen als externe Datei
auch noch übergeben werden kann. Die Ausgabe erfolgt durch das Schreiben der
einzelnen Bytes des Maschinencodes in eine Datei im .coe-Format. Der Grund für
das .coe-Format ist, dass dieses genutzt werden kann, um den RAM-Block der EPU
zu initialisieren und somit das Programm direkt ausgeführt werden kann.

Der Aufbau der Befehle, welche der Assembler versteht, wurde recht simpel
gehalten, um eine leicht zu verstehende Sprache zu erstellen, wobei viele
Ähnlichkeiten zur x86-Assembly (genutzt ist den meisten Desktopcomputern
heutzutage) bestehen, da somit das anfängliche Verstehen der Befehle leichter
wird. Der Aufbau eines Befehls richtet sich nach folgender Regel, wobei in
eckigen Klammern eingerahmte Parameter optional sind bzw. nur bei bestimmten
Befehlen notwendig sind:
\begin{lstlisting}[basicstyle=,numbers=none]
	Mnemonic[.Option] [Operand1] [,Operand2] [,Operand3]
\end{lstlisting}

Um einen Befehl aufzurufen, wird das obere Schema verwendet, was bedeutet, dass
der Befehl selbst mit seiner \textbf{Mnemonic} startet. Das Wort Mnemonic
bedeutet auf Deutsch so viel wie 'Merkspruch'. Es wird meist bei
Assemblersprachen verwendet, um beim Programmieren das Auswendiglernen des
Maschinencodes jedes Befehls zu sparen. An die Mnemonic kann eine Option nach
einem Punkt angehängt werden. Dies ist befehlsspezifisch und muss für jeden
Befehl nachgeschlagen werden. Je nach Befehl werden auch ein bis zwei weitere
Operanden benötigt, welche selbst noch in verschiedenen Formen vorkommen können
(abhängig von der gewählten Option). Um dies zu verdeutlichen, sollen nun ein
paar Beispiele das Verständnis für den Befehlsaufbau gezeigt werden.
\begin{figure}
\begin{lstlisting}[basicstyle=]
	load r1, 0xAFFE
	jmp.i $test

	data:
		.data 0xFF

	test:
		addi r2, r1, 0d10
		load r1, $data
		write.l r1, r2, 0x0
\end{lstlisting}
\caption{Befehlsaufbau - Assembly}
\label{code:instruction_code}
\end{figure}

In Abbildung ~\ref{code:instruction_code} wird zuerst in Zeile 1 eine
hexadezimale Konstante (Präfix '0x' steht für hexadezimal) in das Register R1
geladen. Danach wird ein Sprungbefehl mit der Option 'i' ausgeführt, wobei das
'i' für 'Immediate' bzw. Konstante steht und damit die Adresse, zu welcher
gesprungen werden soll, im Befehl vorhanden ist.  Die Konstant ist hier mit dem
Adressoperator '\$' vorangestellt, welcher die Adresse des \textbf{Labels}, hier
test' (Zeile 7), enthält. Dadurch wird die Ausführung in Zeile 8, also direkt
nach der Definition des Labels, fortgeführt.

Davor ist aber noch ein in Zeile 5 eine sogennante \textbf{Assemblerdirektive}
zu sehen, welche an dem '.' vor der Mnemonic zu erkennen ist.
Assemblerdirektiven sind Anweisungen an den Assembler und werden nicht direkt in
Maschinencode umgewandelt, sondern getrennt behandelt. In diesem Beispiel ist
die Assemblerdirektive 'data' verwendet worden, welche die angegebene Konstante
im ersten Operanden unmittelbar in die Byteausgabe des Maschinencodes übernimmt.
So ist es beispielsweise möglich, dass vordefinierte Variablen bereits einen
Wert beim Start des Programms zugewiesen bekommen können.

Zurück bei der Ausführung des Beispieles in Zeile 8 wird mit dem Befehl 'addi'
eine Konstante zu dem Register R1 (zweiter Operand) eine dezimale Konstante
(Präfix '0d' für dezimale Konstanten) addiert. Das Ergebnis wird im ersten
Operand gespeichert - hier R2. In Zeile 9 wird die Adresse des 'data'-Labels in
das Register R1 geladen und danach in Zeile 10 wird an die Speicheradresse in R1
der Wert im Register R2 geschrieben. Der dritte Operand wird hier nicht benötigt
und daher null gesetzt, kann aber für einen Offset der Speicheradresse dienen.
\pagebreak
\section{Funktionsaufruf}
In diesem Abschnitt soll der Aufruf von Funktonen demonstriert werden. Bei dem
Aufbau wurde eine ähnliche Methode wie beim x86-Befehlssatz verwendet. Die
einfachste Methode des Funktionsaufrufs ist das Aufrufen einer Funktion ohne
Parameter, wie unten in Abbildung~\ref{code:function_struct}  zu sehen ist.
\begin{figure}[htb]
\begin{lstlisting}[basicstyle=]
funktion:
	ret

_start:
	call.i $funktion
\end{lstlisting}
\caption{Funktionsaufruf ohne Parameter}
\label{code:function_struct}
\end{figure}
Dabei wird einfach der Befehl 'call' aufgerufen mit dem ersten Operanden hier
als Adresse zum Label 'funktion'. Nachdem die Funktion mit dem Befehl 'ret'
wieder zurückkehrt, wird automatisch der nächste Befehl nach dem Funktonsaufruf
ausgeführt.

Damit beim Zurückkehren von einer Funktion die Rücksprungadresse für den
nächsten Befehl gefunden werden kann, wird beim Aufruf von 'call' nicht nur ein
Sprungbefehl zur Funktion aufgerufen, sondern auch die Rückkehradresse auf den
Stack (siehe~\ref{s:stack}) gelegt. Beim Ausführen von 'ret' wird dann einfach
das Element vom Stack heruntergenommen und in den PC (siehe~\ref{s:pc})
geschrieben.

Bei Funktionen mit Parametern und Rückgabewert ist der Aufbau ähnlich. Das
folgende Beispiel in Abbildung~\ref{code:function_parameter} soll einen solchen
Funktionsaufruf illustrieren:
\begin{figure}[!htb]
\begin{lstlisting}[basicstyle=]
sub:
	pop r15
	pop r0
	pop r1
	push r15

	sub.u r0, r0, r1

	pop r15
	push r0
	push r15
	ret

_start:
	load r4, 0x0005
	load r5, 0x0002
	push r5
	push r4
	call.i $sub
	pop r3
\end{lstlisting}
\caption{Funktionsaufruf mit Paramtern}
\label{code:function_parameter}
\end{figure}

Zuerst werden die zu übergebenen Argumente in beliebige Register gespeichert
(Zeile 15-16) und dann in \textbf{umgekehrter} Reihenfolge auf den Stack gelegt
(Zeile 17-18). Danach findet der eigentliche Funktionsaufruf statt und die
Funktion wird aufgerufen, wobei unter anderem die Rückkehradresse wird auch auf
dem Stack abgelegt. Da die Rückkehradresse ganz oben auf dem Stack abgelegt ist,
muss diese erst vom Stack heruntergenommen und in einem Register
zwischengespeichert werden, damit die Argumente vom Stack geholt werden können
(Zeile 2). Da sie in umgekehrter Reihenfolge auf den Stack gelegt wurden, dreht
sich beim Holen der Argumente die Reihenfolge wieder um, sodass die Reihenfolge
wieder richtig ist (Zeile 3-4). Danach kann die Rückkehradresse wieder zurück
auf den Stack gelegt werden (Zeile 5). Danach kann die eigentliche Operation der
Funktion durchgeführt werden; dies ist hier die Subtraktion (Zeile 7). Da nun
der Rückgabewert auf dem Stack abgelegt werden soll, aber auch zugriffsbereit
für die aufrufende Funktion, muss zuerst wieder die Rückkehradresse vom Stack
entfernt (Zeile 9) und dann der Rückgabewert auf den Stack gelegt werden (Zeile
10). Danach kann die Rückkehradresse wieder auf den Stack (Zeile 11) und die
Funktion kann wie gewohnt zurückkehren (Zeile 12).

Durch die Nutzung des Stacks für den Rückgabewert ist es theoretisch möglich,
mehrere Rückgabewert einer Funktion zuzuweisen. In dem Beispiel ist dies aber
nicht der Fall.  Dennoch ist zu erwähnen, dass über die Register auch ein oder
mehrere Rückgabewerte theoretisch erfolgen können. Genauere Informationen sind
der jeweiligen Funktion, welche man aufrufen möchte, zu entnehmen.
\clearpage
\section{Vordefinierte Funktonen}
Einige allgemein nützige Funktionen sind bereits vorgeschrieben worden und
können als Vorlage an den Assembler übergeben werden. Dadurch können diese
vordefinierten Funktionen in allen Programmen genutzt werden. Unter anderem sind
folgende Funktionen verfügbar:
\begin{description}
\item [\_print(*data, start)] Gebe eine null-terminierte Zeichenkette an der
Stelle \textbf{data} auf dem Bildschirm an Startpositon \textbf{start}
aus.
\item [\_setcursor(pos)] Setze den Bildschirmcursor an Position \textbf{pos},
	wobei der Wert 0xAABB die x-Koordinate 0xAA und die y-Koordinate 0xBB setzt.
\item [\_getinput] Liest ein Byte vom Benutzer ein und gibt es zurück
\item [\_getstring] Liest einen 'Enter'-terminierten String vom Benutzer ein und
	gibt diesen zurück
\end{description}
Die Benutzung dieser Vorlage wird empfohlen, ist aber keine Voraussetzung. Denn
die Vorlage definiert auch, dass der Start des eigentlichen Programms beim Label
'\_start' stattfindet und dieser wird auch von der Vorlage aufgerufen. Wie
wahrscheinlich bereits aufgefallen ist, starten bei der Erstellung der Vorlage
alle genutzten Labels mit einem Unterstrich, was den Grund hat, alle
vordefinierten Funktionen und die Funktionen des Programms visuell zu trennen.
Daher ist es auch nicht empfohlen, dass im Programm Labels mit einem Unterstrich
starten.
\pagebreak
\section{Debugging}
Mithilfe des Debugging ist es möglich, Fehler in Programmen zu erkennen und
diese auch zu beseitigen. Damit dies auch bei der \ac{EPU} möglich ist, soll
dieser Abschnitt beispielhaft anhand eines Signal-Zeit-Diagrammes zeigen, wie die
Ausführung eines Befehls aussehen sollte.
Das auszuführende Befehl sei folgender, wobei angenommen wird das Register R14
bereits den Wert 0x1234 hat:
\begin{lstlisting}[basicstyle=,numbers=none]
	push r14
\end{lstlisting}

\captionsetup[figure]{justification=centering,singlelinecheck=false}
\begin{figure}[htb]
\raggedright
\begin{wave}{11}{9}{0}
 \nextwave{en\_fetch} \bit{1}{1} \bit{0}{1} \bit{1}{1} \bit{0}{1} \bit{1}{1}
					 \bit{0}{5}
 \nextwave{en\_decode} \bit{0}{1} \bit{1}{1} \bit{0}{1} \bit{1}{1} \bit{0}{1}
					   \bit{1}{1} \bit{0}{4}
 \nextwave{en\_regread} \bit{0}{6} \bit{1}{1} \bit{0}{3}
 \nextwave{en\_alu} \bit{0}{7} \bit{1}{1} \bit{0}{2}
 \nextwave{en\_stack} \bit{0}{8} \bit{1}{1} \bit{0}{1}
 \nextwave{en\_regwrite} \bit{0}{9} \bit{1}{1}
 \nextwave{pc} \known{0x0}{2} \known{0x1}{2} \known{0x2}{6}
 \nextwave{dataA} \unknown{7} \known{0x1234}{3}
 \nextwave{stack\_i\_en} \bit{0}{8} \bit{1}{1} \bit{0}{1}
 \nextwave{stack\_i\_we} \bit{0}{8} \bit{1}{1} \bit{0}{1}
\end{wave}
\caption{Signal-Zeit-Diagramm}
\label{wave:program}
\end{figure}

Wie in Abbildung~\ref{wave:program} zu sehen, beschreiben die ersten 6 Signale
die einzelnen Zustände des Steuerwerks (siehe~\ref{s:control}), danach folgt der
Programmzähler (siehe~\ref{s:pc}) und der Wert des übergebenen Registers Ra,
gefolgt von dem 'Enable'- und 'WriteEnable'-Eingang des Stacks
(siehe~\ref{s:stack}).

Im ersten Abschnitt von t=0 bis t=6 ist zu sehen, wie byteweise der Befehl
gelesen und dekodiert wird. Dabei wird bei jeder positiven Flanke des Signals
'en\_fetch' der Programmzähler inkrementiert, damit er auf das nächste zu
dekodierende Byte zeigt. Im zweiten Abschnitt von t=6 bis t=9 werden zuerst die
Werte der übergebenen Register gelesen und sind dann nach der negativen Flanke
von 'en\_regread' verfügbar (siehe Signal 'dataA'). Als nächstes wird das
Rechenwerk aktiviert und legt in diesem Fall nur fest, dass kein Sprungbefehl
ausgeführt werden muss.Danach wird nun 'en\_stack' auf 1 gesetzt (t=8) und damit
gleichzeit auch die 'Enable'- und 'Write-Enable'-Eingänge des Stacks. Der Stack
nimmt daraufhin den Wert von 'dataA' und schreibt ihn als oberstes Element auf
den Stapel. Im letzten Schritt (t=9) wird nun noch 'en\_regwrite', da jedoch
kein Ergebnis in ein Register geschrieben werden muss, passiert hier nichts.
